\section{Radiative Processes}

Remember guys, there's nothing funny about neutrinos.

\subsection{Questions}

\begin{enumerate}

\item \textbf{Derive the total power and characteristic frequency of synchrotron radiation from
      a relativistic particle of mass m, charge e, and energy E moving in a magnetic field B.
      Use this to explain why synchrotron radiation is generally negligible for protons.}
      
      $\bullet$ Synchrotron radiation is caused by relativistic charged particles moving in an external magnetic field. Therefore, the Lorentz force equations of motion and the Larmor formula for the power radiated by an accelerating charged particle will be relevant. 
      
      Lorentz force: 
      \begin{equation}
      F_{B} = \frac{q}{c} \bar{v} \times \bar{B}
      \end{equation}
      
      Larmor Formula:
      \begin{equation}
      P = \frac{2}{3} \frac{q^{2}}{c^{3}} \gamma^{4} a^{2}
      \end{equation}
      
      $\bullet$ Use the equations of motion to solve for the acceleration $a$ of the particle and plug into the Larmor formula to get the total power radiated from a single particle.
      
      \begin{equation}
      F_{B} = \frac{d}{dt} (\gamma m \bar{v})
      \end{equation}
      
      We can determine that $\gamma$ is constant by using the definition of work and power.
      
      \begin{equation}
      \frac{dE}{dt} = \frac{d}{dt} (\bar{F} \cdot distance) = \bar{F} \cdot \frac{dx}{dt} = \bar{F} \cdot \bar{v}
      \end{equation}
      
      But since $F_{B}$ is perpendicular to $\bar{v}$, this whole equation is $0$. This implies $E = \gamma mc^{2}$ is constant, so $\gamma$ is constant. This simplifies our equation of motion:
      
      \begin{equation}
      \gamma m \frac{d\bar{v}}{dt} = \frac{q}{c} \bar{v} \times \bar{B}
      \end{equation}
      
      Now the acceleration can be solved for directly. In the direction of B, there is no acceleration (remember the cross product). Perpendicular to B:
      
      \begin{equation}
      \frac{dv}{dt}_{\perp} = \frac{q}{\gamma mc} vB \sin{\alpha}
      \end{equation}
      
      where $\alpha$ is the angle between the velocity and the magnetic field. Plug into the Larmor Formula to get the total power radiated by a single electron:
      
      \begin{equation}
      P = \frac{2}{3} \frac{q^{4}B^{2}}{c^{5}m^{2}} \gamma^{2} v^{2} \sin{\alpha}^{2}
      \end{equation} 
      
      $\bullet$ The emission from the synchrotron has a characteristic frequency because it gyrates with a certain frequency, and is also beamed because of its relativistic motion. The emission is confined to a cone of radius $\frac{2}{\gamma}$. Observers can only see the emission during the times this cone is pointed in the line of sight. Solve for the time the particle spends in this section of its orbit:
      
      \begin{equation}
      \Delta t = \frac{\Delta s}{v} = \frac{2r}{\gamma v}
      \end{equation}
      
      Use the equation of motion to solve for either the arc length $\Delta s$ or radius of curvature $r$.
      
      \begin{equation}
      \Delta t = \frac{2mc}{qB\sin{\alpha}}
      \end{equation}
      
      $\bullet$ What is seen is the light emitted from this region of space, so subtract the light travel time from the beginning to the end of the path to get the duration of the pulse:
      
       \begin{equation}
      \Delta t^{A} = \frac{2mc}{qB\sin{\alpha}} - \frac{\Delta s}{c} = \Delta t (1 - \frac{v}{c})
      \end{equation}
      
      \begin{equation}
      (1 - \frac{v}{c}) \approx \frac{1}{2\gamma^{2}}
      \end{equation}
      
      If you write out the definition of $\gamma$ and factor the polynomial, you can approximate $(1 + \frac{v}{c}) \approx 2$ since the particle is very relativistic.
      
      $\bullet$ The inverse of this is the characteristic or 'critical' frequency. It tells you the spectrum will be centered around this frequency.
      
      $\bullet$ A proton is $\sim10^{4}$ times as massive as an electron. To compare electron and proton synchrotron radiation, think about which quantities are affected by mass. Power is inversely proportional to mass, so it won't radiate with as much power for a proton. Also, for a given energy, the proton won't move as fast, lowering $\gamma$. This lowers the power as well, but it also makes it harder for the proton to reach synchrotron energies at all. Finally, the characteristic frequency is proportional to $\gamma$, so a proton's peak frequency will be at longer wavelengths, therefore with lower energy.
      
\item \textbf{Explain the connection between detailed balance, the Einstein A and B relations,
      Kirchoff's law and the Milne relations, and give an example of their use to connect the
      bremsstrahlung emission spectrum and the free-free absorption coefficient.}
      
      Einstein relations:
      
      Ignore collisions for now and just think about particles that are radiating and absorbing photons. Consider a gas with $n_1$ number density in the lower state (1) and $n_2$ in the excited state (2). Assume thermodynamic equilibrium for now. We have Einstein relations
      
      $A_{21}$: probability of spontaneous emission per time (s$^{-1}$),
      
      $B_{12}\bar{J}$: probability of absorption per time (s$^{-1}$),
      
       $B_{21}\bar{J}$: probability of stimulated emission per time (s$^{-1}$).
       
       $\bar{J}$ is just the mean intensity averaged over the line profile for the transition:
       \begin{equation}
       \bar{J} = \int^\infty_0 J_\nu \phi(\nu) d\nu \,\,.
       \end{equation}
       
       Also recall the Maxwell-Boltzmann relationship between $n_1$ and $n_2$:
       \begin{equation}
       \frac{n_1}{n_2} = \frac{g_1}{g_2} e^{h\nu_0 / k_B T},
       \end{equation}
       where $h\nu_0$ is the energy of the photon and the energy difference between the two levels, $E_2 - E_1$. You can use these equations to solve for $\bar{J}$, then equate $\bar{J}$ to $B_\nu$ assuming $J_\nu = B_\nu$ and $B_\nu$ varies very little over the sharp line profile $\phi(\nu)$. Equating these two gives the following Einstein relations:
      
      \begin{equation}
      \frac{g1}{g2} = \frac{B_{21}}{B_{12}}
      \end{equation}
      
      \begin{equation}
      \frac{A_{21}}{B_{21}} = \frac{2 h \nu^3}{c^2}
      \end{equation}
      
      Even though we assumed thermodynamic equilibrium, these relations do not depend on temperature and just describe properties of the atoms. Therefore, we can say they are more general and hold for cases that are not in thermodynamic equilibrium.
      
      Kirchoff's law:
      
      \begin{equation}
      j_\nu = \alpha_\nu B_\nu.
      \end{equation}
      
      Show the following:
      
      \begin{equation}
      j_\nu = \frac{h \nu}{4 \pi} n_2 A_{21} \phi(\nu)
      \end{equation}
      
       \begin{equation}
      \alpha_\nu = \frac{h \nu}{4 \pi} (n_1 B_{12} - n_2 B_{21} ) \phi(\nu)
      \end{equation}
      
      (see p. 5 of Tony's notes). 
      
      Milne relations:
      
      These relations are analogous to the Einstein relations, but they are for photoionization and radiative recombination, rather than transitions between energy levels in an atom. Following Rybicky \& Lightman (see p.\,284 for more discussion), assume we have a thermal velocity distribution of electrons and that the radiation field is described by the Planck function $B_\nu$. Let $\sigma_{\rm fb}(v)$ be the cross section for an electron with speed $v$ to recombine with an ion; meanwhile, $\sigma_{\rm bf}(\nu)$ is the cross section for an ion to to be photoionized by a photon with frequency $\nu$. We can write the rate of recombinations per volume (s$^{-1}$ cm$^{-3}$) in terms of the number densities of ions and electrons, the cross section, and the electron velocity distribution for a velocity range $dv$:
      
      \begin{equation}
      R_{\rm recomb} = N_+N_e \sigma_{\rm fb} f(v) dv
      \end{equation}
      
      and the rate of ionization is
      
       \begin{equation}
      R_{\rm ioniz} = \frac{4\pi}{h\nu} N_n \sigma_{\rm bf} (1 - e^{-h\nu/k_{\rm B}T}) I_\nu d\nu
      \end{equation}
      
      and we have assumed $I_\nu = B_\nu$ (note: where does the factor of $(1 - e^{-h\nu/k_{\rm B}T})$ come from? Also, justify these expressions to yourself in general).
      
      Set these two equations equal and obtain an expression for $\sigma_{\rm bf} / \sigma_{\rm fb}$; use the Maxwell-Boltzmann distribution for $f(v)$ and the Saha equation to obtain the Milne relation:
      
      \begin{equation}
      \frac{\sigma_{\rm bf}}{\sigma_{\rm fb}} = \frac{m^2 c^2 v^2 g_e g_+}{\nu^2 h^2 2 g_n}\,\, .
      \end{equation}
      
      The Einstein relations, Milne relations, and Kirchoff's law all describe relationships between the absorption and emission of photons. Einstein deals with transitions between two energy levels on an atomic level using the properties and thermal distributions of atoms. Milne is the analogous relationship for ionization and radiative recombination. The Kirchoff law is general and describes both in terms of the bulk properties of the material like emission and absorption coefficients (io is kind of bullshitting here. Does anyone have a better explanation for tying these together?).
      
      
\item \textbf{Draw the energy levels of the hydrogen atom and identify which transitions are
      allowed. Which ones are in the visible part of the spectrum? Which level has no
      allowed decays, and what is its main decay mode?}
      
\end{enumerate}
