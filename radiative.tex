\section{Radiative Processes}

Remember guys, there's nothing funny about neutrinos.

\subsection{Questions}

\begin{enumerate}

\item \textbf{Derive the total power and characteristic frequency of synchrotron radiation from
      a relativistic particle of mass m, charge e, and energy E moving in a magnetic field B.
      Use this to explain why synchrotron radiation is generally negligible for protons.}
      
      \newthought{Synchrotron radiation is}
      caused by relativistic charged particles moving in an external
      magnetic field. Therefore, the Lorentz force equations of motion and the Larmor formula
      for the power radiated by an accelerating charged particle will be relevant. 
      \begin{dgroup}
      \begin{dmath}
        \text{Lorentz Force:}\quad F_{B} = \frac{q}{c} \v{v} \times \v{B}
      \end{dmath}
      \begin{dmath}
        \text{Larmor Formula:}\quad P = \frac{2}{3} \frac{q^{2}}{c^{3}} \gamma^{4} a^{2}
      \end{dmath}
      \end{dgroup}
      Use the equations of motion to solve for the acceleration, $a$, of the particle and plug
      into the Larmor formula to get the total power radiated from a single particle.
      \begin{dgroup*}
      \begin{dmath*}
            F_{B} = \frac{\d}{\d t} (\gamma m \v{v})
      \end{dmath*}
      \begin{dmath*}
            \frac{q}{c} \v{v} \times \v{B} = \gamma m \frac{\d\v v}{\d t},
      \end{dmath*}
      \end{dgroup*}
      where we have used the fact that $\gamma$ is constant\sidenote{
        From the definition of work:
        \begin{dgroup*}
        \begin{dmath*}
        \d W = \v F\cdot \v{\d s}
        \end{dmath*}
        \end{dgroup*}
        However, the Lorentz force $F_B$ is always perpendiculr to $\v{\d s}$ so that
        no work is done on the particle.  Therefore the speed (and hence $\gamma$) is constant.
      }.
      
      Now the acceleration can be solved for directly. In the direction of $\v B$,
      there is no acceleration (remember the cross product). Perpendicular to $\v B$:
      \begin{equation}
        \left(\frac{\d v}{\d t}\right)_{\perp} = \frac{q}{\gamma mc} vB \sin{\alpha},
      \end{equation}
      where $\alpha$ is the angle between the velocity and the magnetic field.
      Plug into the Larmor Formula to get the total power radiated by a single electron:
      \begin{equation}\boxed{
        P = \frac{2}{3} \frac{q^{4}B^{2}}{c^{5}m^{2}} \gamma^{2} v^{2} \sin{\alpha}^{2}
      }\end{equation} 
      
      \newthought{The characteristic frequency} of synchrotron emission is influenced
      by two effects: the gyro-frequency\sidenote{
        The frequency at which a relativistic electron gyrates in a constant magnetic field.
      }, and relativistic beaming.

      The opening angle of the cone of emission (due to the beaming) is $\sim2/\gamma$.
      Therefore an observer can only see the emission during the time it takes the electron
      to travel $2/\gamma$ radians along its circular orbit.  This corresponds to an arclength
      of $2r/\gamma$ where $r$ is the Larmor radius\sidenote{
        The radius of an electron's orbit in a uniform magnetic field.  Begin by equating
        the Lorentz force with the centripetal force:
        \begin{dgroup*}
        \begin{dmath*}
            \frac{qvB}{c}\sin\alpha = \frac{\gamma mv^2}{r}
        \end{dmath*}
        \begin{dmath*}
            r = \frac{\gamma mvc}{qB\sin\alpha}
        \end{dmath*}
        \end{dgroup*}
      }.
      Therefore, if the electron is moving with a speed $v$, it is only visible for a time
      \begin{dmath}
        \Delta t = \frac{2r}{\gamma v}\nolinebreak=\nolinebreak \frac{2mc}{qB\sin\alpha}
      \end{dmath}.
      However, the light emitted at the beginning of the pulse has to travel an extra
      distance $2r/\gamma$ (assuming $\gamma \gg 1$ we can treat the arc as a straight line).
      The distance in arrival time between the beginning and the end of the pulse is then
      \begin{dmath}
        \Delta t^\star = \Delta t - \frac{2r}{\gamma c}
            \nolinebreak=\nolinebreak \Delta t\left(1-\frac{v}{c}\right)
            \nolinebreak\approx\nolinebreak \frac{\Delta t}{2\gamma^2}
      \end{dmath},
      where the approximation\sidenote{
        \begin{dmath*}
            \frac{1}{\gamma^2} = 1-\frac{v^2}{c^2} = \left(1-\frac{v}{c}\right)\left(1+\frac{v}{c}\right)
                \approx 2\left(1-\frac{v}{c}\right)
        \end{dmath*}
      } requires $\gamma \gg 1$.

      The Fourier transform of the pulse shape gives the frequency structure of the synchrotron
      emission, however, because the pulse width is $\approx\Delta t^\star$, we the largest
      contribution to the Fourier integral will be modes with $\nu\sim1/\Delta t^\star$.
      Therefore we define the critical frequency to be
      \begin{dmath}\boxed{
        \nu_c \sim \frac{1}{\Delta t^\star} \nolinebreak=\nolinebreak \frac{\gamma^2 qB\sin\alpha}{mc}
      }\end{dmath}.

      The question asks for the synchrotron spectrum as a function of energy (not $\gamma$).
      Simply set $\gamma = E/mc^2$ in the above equations to eliminate $\gamma$.  Finally note
      that protons are $\sim 1800$ times more massive than electrons.

\item \textbf{Explain the connection between detailed balance, the Einstein A and B relations,
      Kirchoff's law and the Milne relations, and give an example of their use to connect the
      bremsstrahlung emission spectrum and the free-free absorption coefficient.}
      
      \newthought{Let's begin with} detailed balance\sidenote{
        Any ideas on what distinguishes detailed balance from un-detailed balance?
      }.  Detailed balance is simply a statement that there are an equal number of transitions
      into and out of a given state.
      \begin{dmath}
        \text{rate of transitions into state A} = \text{rate of transitions out of state A}
      \end{dmath}
      Einstein's coefficients tell us how to express this idea mathematically
      for radiative transitions\sidenote{
        Transitions that involve the emission or absorption of a photon.
      }.
      Consider a two state system where state (1) has a lower energy than state (2).
      Given an incident mean intensity (averaged over the line profile\sidenote{
        The mean intensity is
        \begin{dmath*}
            J_\nu\equiv\frac{1}{4\pi}\int I_\nu\d\Omega
        \end{dmath*},
        and averaging over the line profile gives us
        \begin{dmath*}
            \bar J = \int J_\nu\phi(\nu)\d\nu
        \end{dmath*}.
      }),
      $A_{21}$ is the probability of spontaneous emission per unit time,
      $B_{12}\bar J$ is the probability of absorption per unit time, and
      $B_{21}\bar J$ is the probability of stimulated emission per unit time.
      Detailed balance therefore tells us (for this two-state system) that
      \begin{dmath}\label{eq:detailed_balance}
        n_2A_{21} = (n_1B_{12}-n_2B_{21})\bar J
      \end{dmath}.
      In thermodynamic equilibrium (and with Maxwell-Boltzmann statistics\sidenote{
        This means spin-statistics aren't important.  We aren't dealing with bosons
        or fermions here.
      }), we have
       \begin{dmath}
        \frac{n_1}{n_2} = \frac{g_1}{g_2} e^{h\nu_0 / k_B T}
       \end{dmath},
       where $h\nu_0$ is the energy of the photon and the energy difference between the two levels,
       $E_2 - E_1$.  Similarly, in thermodynamic equilibrium (with a narrow line profile), the
       line-averaged mean intensity is given by the Planck function\sidenote{
        \begin{dmath*}
            \bar J = B_\nu\nolinebreak= \frac{2h\nu^3}{c^2}\frac{1}{e^{h\nu/kT}-1}
        \end{dmath*}
       }.  These two relations combined with the expression for detailed balance given
       in Equation~\ref{eq:detailed_balance} imply that
       \begin{dgroup}
       \begin{dmath}
        g_1B_{12} = g_2B_{21}
       \end{dmath},
       \begin{dmath}
        A_{21} = \frac{2 h \nu^3}{c^2}B_{21}
       \end{dmath}.
       \end{dgroup}
       These are called the Einstein relations.
      Even though we assumed thermodynamic equilibrium, these relations do not depend on temperature and just describe properties of the atoms. Therefore, we can say they are more general and hold for cases that are not in thermodynamic equilibrium.
      Also, the A and B coefficients are intrinsic properties of individual 
      atoms, so they should have nothing to do with the overall state of the 
      gas and whether or not it's in thermodynamic equilibrium.

      \newthought{Now let's discuss} Kirchoff's law.
      Take a thermally emitting (not necessarily blackbody) material at temperature T and place it 
      in a blackbody cavity also at temperature T.  With just the cavity, $S_{\nu}=I_{\nu}=B_{\nu}$.  
      With the material added, we still have a cavity of temperature T, so the total intensity must 
      still be a blackbody, since by definition this intensity does not depend on the internal 
      structure.  The added material has some source function $S_{\nu 2}$ that affects the total 
      intensity of the cavity.  Think of the new material as a new piece of the cavity, with radiation 
      contributed by its source function.  If the source function is greater than the blackbody function 
      of the cavity, too much radiation will be contributed and the resulting intensity will also be 
      greater than the blackbody of the cavity on its own.  If the source function is less than the 
      blackbody function, too little radiation will be contributed, decreasing the total intensity to 
      below the blackbody function.  But we know the total intensity must still be a blackbody function 
      of temperature T.  So the source function of the new material must be a blackbody of temperature T.  
      So, we have 
      \begin{displaymath}S_{\nu}=B_{\nu}\end{displaymath}
      and, by definition of $S_{\nu}$,
      \begin{equation}
      j_\nu = \alpha_\nu B_\nu.
      \end{equation}
      To clarify, the source function of a material in thermodynamic equilibrium is the
      Planck function.
      
      We can relate the Einstein coefficients to the emission and absorption coefficients
      by noting that each transition adds or removes $h\nu$ in energy spread over a solid
      angle of $4\pi$.  Therefore
      \begin{dgroup}
      \begin{dmath}
      j_\nu = \frac{h \nu}{4 \pi} n_2 A_{21} \phi(\nu)
      \end{dmath},
      \begin{dmath}
      \alpha_\nu = \frac{h \nu}{4 \pi} (n_1 B_{12} - n_2 B_{21} ) \phi(\nu)
      \end{dmath}.
      \end{dgroup}
      
      Milne relations:
      
      These relations are analogous to the Einstein relations, but they are for photoionization and radiative recombination, rather than transitions between energy levels in an atom. Following Rybicky \& Lightman (see p.\,284 for more discussion), assume we have a thermal velocity distribution of electrons and that the radiation field is described by the Planck function $B_\nu$. Let $\sigma_{\rm fb}(v)$ be the cross section for an electron with speed $v$ to recombine with an ion; meanwhile, $\sigma_{\rm bf}(\nu)$ is the cross section for an ion to to be photoionized by a photon with frequency $\nu$. We can write the rate of recombinations per volume (s$^{-1}$ cm$^{-3}$) in terms of the number densities of ions and electrons, the cross section, and the electron velocity distribution for a velocity range $dv$:
      
      \begin{equation}
      R_{\rm recomb} = N_+N_e \sigma_{\rm fb} f(v) dv
      \end{equation}
      
      and the rate of ionization is
      
       \begin{equation}
      R_{\rm ioniz} = \frac{4\pi}{h\nu} N_n \sigma_{\rm bf} (1 - e^{-h\nu/k_{\rm B}T}) I_\nu d\nu
      \end{equation}
      
      and we have assumed $I_\nu = B_\nu$ (note: where does the factor of $(1 - e^{-h\nu/k_{\rm B}T})$ come from? Also, justify these expressions to yourself in general).
      
      Set these two equations equal and obtain an expression for $\sigma_{\rm bf} / \sigma_{\rm fb}$; use the Maxwell-Boltzmann distribution for $f(v)$ and the Saha equation to obtain the Milne relation:
      
      \begin{equation}
      \frac{\sigma_{\rm bf}}{\sigma_{\rm fb}} = \frac{m^2 c^2 v^2 g_e g_+}{\nu^2 h^2 2 g_n}\,\, .
      \end{equation}
      
      The Einstein relations, Milne relations, and Kirchoff's law all describe relationships between the absorption and emission of photons. Einstein deals with transitions between two energy levels on an atomic level using the properties and thermal distributions of atoms. Milne is the analogous relationship for ionization and radiative recombination. The Kirchoff law is general and describes both in terms of the bulk properties of the material like emission and absorption coefficients (io is kind of bullshitting here. Does anyone have a better explanation for tying these together?).
      The question is a little vague, I think Io's answer is pretty good.  
      Kirchoff's law and the Einstein relations both relate emission and 
      absorption of photons, but Kirchoff's law only holds for thermal 
      radiation.  The Einstein relations are more general and extend Kirchoff's 
      law to non-thermal radiation processes.  I don't really 
      see a relationship like that between the Einstein and Milne relations, 
      so maybe the question just wants us to say the connection between 
      detailed balance and the Einstein relations and the connection between 
      detailed balance and the Milne relations.
      
      Bremsstrahlung:\newline
      Kirchoff's law can be used to determine the free-free absorption coefficient.  Free-free 
      absorption is just the inverse process to Bremsstrahlung emission, so by knowing the 
      emission coefficient from Bremsstrahlung, the absorption coefficient can be calculted using 
      Kirchoff's law:
      \begin{displaymath}j_{\nu}=\alpha_{\nu}B_{\nu}\end{displaymath}
      The emission coefficient can be derived by treating Bremsstrahlung interactions as Thompson 
      scattering by virtual photons in the frame of the electron.  For a thermal population 
      of electrons, this gives an expression for the emission coefficient of 
      \begin{displaymath}\epsilon_{\nu}^{ff}=6.8\times10^{-38}T^{-1/2}Z^2n_en_ie^{-h\nu/kT}\overline{g_{ff}}ergs/s/cm^3/Hz\end{displaymath}
      Using $j_{\nu}^{ff}=\epsilon_{\nu}^{ff}/4\pi$ and the expression for the Blackbody function, and 
      plugging into Kirchoff's law, gives:
      \begin{displaymath}\alpha_{\nu}^{ff}=3.7\times10^8T^{-1/2}Z^2n_en_i\nu^{-3}(1-e^{-h\nu/kT})\overline{g_{ff}}cm^{-1}\end{displaymath}

      
\item \textbf{Draw the energy levels of the hydrogen atom and identify which transitions are
      allowed. Which ones are in the visible part of the spectrum? Which level has no
      allowed decays, and what is its main decay mode?}

      \begin{table}[ht]
      \begin{tabular}{cccc}
      $\vdots$ & $\vdots$ & $\vdots$ & $\vdots$ \\
      4s & 4p & 4d & 4f \\
      3s & 3p & 3d &\\
      2s & 2p &&\\
      1s &&&\\
      \end{tabular}
      \caption{The energy levels of the hydrogen atom (ignoring fine structure).  The first
               number is the principal quantum number, usually denoted by $n$.  The second letter
               denotes the orbital angular momentum $l$, where s, p, d, and f denote
               $l=1,\,2,\,3,\,4$ respectively.  The energy of a state is given by
               $E_n = -13.6\,{\rm eV}/n^2$.}
      \label{tab:hydrogen_energy_levels}
      \end{table}

      \newthought{The allowed transitions}
      between states are governed by so-called ``selection rules.''
      This, however, is a misnomer because every transition is allowed, some are just strongly
      disfavored over others.  The most likely transitions (the ``allowed transitions'') are
      the electric dipole transitions which follow the following rules (in a one electron atom):
      \begin{dgroup}
      \begin{dmath}
        \Delta l = \pm 1
      \end{dmath}
      \begin{dmath}
        \Delta m = 0,\,\pm1
      \end{dmath}
      \end{dgroup}
      Keeping these selection rules in mind, and looking at Table~\ref{tab:hydrogen_energy_levels},
      the only state that has no allowed decays is the 2s state.  Its main decay mode is through
      the spontaneous emission of two photons.
      
      As far as optical transitions go, the 4 lowest energy Balmer transitions (transitions
      to $n=2$) are visible.  Table~\ref{tab:spectral_lines} gives a little more detail.

\end{enumerate}

\subsection{Specific intensity}
Specific intensity $I_{\nu}$ = energy per time per area per frequency per
steradian (erg/s/Hz/ster/cm$^2$).

\begin{table}[ht]
\begin{tabular}{cc}
\hline\hline
Specific mean intensity (erg/s/cm$^2$/Hz) &
    $J_{\nu}=\frac{1}{4\pi}\int I_{\nu}\,d\Omega$ \\
Specific energy density (erg/cm$^3$/Hz) &
    $u_{\nu}=\frac{4\pi}{c}J_{\nu}$ \\
Energy Flux (erg/s/cm$^2$/Hz) &
    $F_{\nu}=\int I_{\nu}\cos \theta\,d\Omega$ \\
Momentum Flux &
    $p_{\nu}=\frac{1}{c}\int I_{\nu}\cos^2 \theta\,d\Omega$ \\
\hline\hline
\end{tabular}
\caption{Moments of the specific intensity}
\end{table}

\subsection{Radiative Transfer}
In steady state, with no scattering, along a path s, change in intensity is emission - absorption.
\begin{dmath}\frac{dI_{\nu}}{ds}=j_{\nu}-\kappa_{\nu}I_{\nu}\end{dmath}
where $\kappa_{\nu}$=absorption opacity=$n\sigma$=$\frac{1}{\lambda_{\nu}}$ and $j_{\nu}$ is 
defined as the energy emitted per unit volume per steradian per second per Hz.  This is similar 
to emissivity, except emissivity ($\epsilon_{\nu}$) is per unit mass, not per unit volume.
An alternate for of the radiative transfer equation is 
\begin{dmath}\frac{dI_{\nu}}{d\tau_{\nu}}=-I_{\nu}+S_{\nu}\end{dmath}
$S_{\nu}=\frac{j_{\nu}}{\kappa_{\nu}}$ is the source function and $d\tau_{\nu}=\kappa_{\nu}ds$ 
is measured along the path of the light ray (from the source to the observer).  If optical depth 
is measured from the observer to the source, $d\tau_{\nu}$ picks up a negative sign, and the 
equation becomes
\begin{dmath}\frac{dI_{\nu}}{d\tau_{\nu}}=I_{\nu}-S_{\nu}\end{dmath}

Solutions to the radiative transfer equation:
\begin{dgroup}
\begin{dmath}
\text{Constant source function:}\quad
    I_{\nu}(\tau_{\nu})=I_{\nu}(0)e^{-\tau_{\nu}}+S_{\nu}(1-e^{-\tau_{\nu}})
\end{dmath}
\begin{dmath}
\text{Pure absorption:}\quad
    I_{\nu}(s)=I_{\nu}(0)e^{-\tau_{\nu}}\nolinebreak=I_{\nu}(0)e^{-\kappa_{\nu} s}
\end{dmath}
\begin{dmath}
\text{Pure emission:}\quad
    I_{\nu}(s)=I_{\nu}(0)+\int_0^s j_{\nu}(s')\,ds'
\end{dmath}
\end{dgroup}

Random Walk:\newline
Good derivation in Carrol and Ostlie of 3-D random walk.  Number of steps needed to travel a 
distance R is $R^2\lambda^2$.  Diffusion time is thus $\frac{R^2}{\lambda c}$ or $\frac{\tau R}{c}$ 
for constant optical depth.  Technically, I think there should be a factor of 3 the numerator, 
but this is all handwavy and order of magnitude anyway.

\subsection{Blackbody Emission}
Derivation of the Blackbody Function:
Start with specific intensity.  Specific intensity is energy per time per area per frequency per 
steradian.  Consider photons moving through an area dA in a direction so that they continue 
through a solid angle d$\Omega$ (there's a drawing in our notes).  The energy these photons carry 
is $h\nu dN$, where $dN=f_{\nu}(\v{x},\v{p})d^3xd^3p$.  $d^3x$ is just the volume a photon can be in 
and still pass though dA at an angel we want, so this is just $cdt\cos{\theta}dA$.  $d^3p$ is the 
volume in momentum space a photon can be in.  It must be travelling in a direction to pass through 
$d\Omega$, so the region of momentum space it can be in is also limited by $d\Omega$.  So 
$d^3p=d\Omega p^2dp=d\Omega \frac{h^2\nu^2}{c^2}\frac{h}{c}d\nu$.  Putting this all together gives 
\begin{displaymath}dE=\frac{h^2\nu^2}{c^2}\frac{h}{c}h\nu c\cos{\theta}d\nu d\Omega dtdAf_{\nu}\end{displaymath}
\begin{displaymath}dE=\frac{h^4\nu^3}{c^2}f_{\nu}\cos{\theta}dAd\Omega d\nu dt\end{displaymath}
So $I_{\nu}=\frac{h^4\nu^3}{c^2}f_{\nu}$.
Using the Bose-Einstein distribution for photons, 
\begin{displaymath}f_{\nu}=\frac{2}{e^{\frac{h\nu}{kT}}-1}\end{displaymath}
gives
\begin{displaymath}\boxed{B_{\nu}(T)=\frac{2h\nu^3}{c^2}\frac{1}{e^{\frac{h\nu}{kT}}-1}}\end{displaymath}

Energy density:\newline
Use the moment formula above for u, and integrate $B_{\nu}$ over all frequencies to get 
\begin{displaymath}u=aT^4\end{displaymath}
Limits:
Rayleight-Jeans limit-$h\nu<<kT$
\begin{displaymath}I_{\nu}(T)=\frac{2\nu^2kT}{c^2}\end{displaymath}
Here, T is often used as the brightness temperature.
Wien limit-$h\nu>>kT$
\begin{displaymath}I_{\nu}(T)=\frac{2h\nu^3}{c^2}e^{-\frac{h\nu}{kT}}\end{displaymath}
Wien Displacement Law:
\begin{displaymath}h\nu_{max}=2.82kT\end{displaymath}
\begin{displaymath}T\lambda_{max}=0.29cmK\end{displaymath}
Radiative diffusion:
In the radiative transfer involving thermal radiation, the source function $S_{\nu}=B_{\nu}$.  
This is derived from Kirchoff's Law.  
So the transfer equation becomes 
\begin{displaymath}\frac{dI_{\nu}}{d\tau_{\nu}}=-I_{\nu}+B_{\nu}\end{displaymath}
In the limit of Rayleigh-Jeans limit, 
\begin{displaymath}I_{\nu}(T_b)=\frac{2\nu^2kT_b}{c^2}\end{displaymath}
by the definition of brightness temperature.  Also, 
\begin{displaymath}B_{\nu}(T)=\frac{2\nu^2kT}{c^2}\end{displaymath}
So the transfer equation becomes
\begin{displaymath}\frac{dT_b}{d\tau_{\nu}}=-T_b+T\end{displaymath}
The solution is similar to the intensity solution:
\begin{displaymath}T_b(\tau_{\nu})=T_b(0)e^{-\tau_{\nu}}+T(1-e^{-\tau_{\nu}})\end{displaymath}
This is basically saying for a material of temperature T that is emitting thermally, if the 
material is optically thin, the observed brightness temperature will change based on the amount 
of the material present.  More material means more emission, but the emission is not re-absorbed 
by the extra material.  For an optically thick material, $\tau$ is large, and the observed 
brightness temperature approaches the actual temperature of the material.  In other words, 
optically thick things emit like blackbodies.  The equation for $T_b$ above is just a mathematical 
statement of this.

\subsection{Thomson Scattering}
The cross section is defined by
\begin{dmath*}
\sigma_T = \frac{\text{power radiated by accelerated electron}}{\text{flux incident on the electron}}
\end{dmath*}.
Starting from the dipole approximation, the power radiated by the accelerated electron is given by
\begin{dmath}
P = \frac{2{\ddot d}^2}{3c^3}
\end{dmath}.
By balancing forces ($eE_0sin(\omega t) = m_e\ddot r$)\sidenote{
    Note that we take the time average here so that $\langle \sin^2 \rangle = 1/2$.
}, the power becomes
\begin{dmath}
P = \frac{e^4E_0^2}{3c^3m_e^2}
\end{dmath}.
The flux incident on the electron is given by the time-averaged poynting vector
\begin{dmath}
\langle S\rangle = \frac{cE_0^2}{8\pi}
\end{dmath}.
So finally the Thomson cross-section is given by
\begin{dmath}\boxed{
\sigma_T = \frac{8\pi e^4}{3c^4m_e^2}
}\end{dmath}.

\subsection{Bremsstrahlung}
Let's begin by imagining an electron traveling (in a straight line) by a nucleus
with atomic number $Z$.  The transverse acceleration is
\begin{dmath}
    a_\perp = \frac{Ze^2}{m_e}\frac{1}{x^2+b^2}\frac{b}{\sqrt{x^2+b^2}}
\end{dmath},
The total perturbation to the perpendicular velocity
is therefore given by
\begin{dgroup*}
\begin{dmath*}
    \delta v_\perp = \int_{-\infty}^{+\infty} a_\perp \frac{\d x}{v}
                   = \frac{Ze^2b}{m_ev}\int_{-\infty}^{+\infty} \frac{\d x}{(x^2+b^2)^{3/2}}
\end{dmath*}
\begin{dmath}
    \p{\delta v_\perp}
                   = \frac{2Ze^2}{bm_ev}
\end{dmath}.
\end{dgroup*}
From this point, it is assumed that the acceleration is essentially instantaneous.
By Fourier transforming the Larmor formula, we find\sidenote{
    Dropping the numerical constants from this point forward because this derivation is
    approximate anyway.
}
\begin{dgroup*}
\begin{dmath*}
    P(t) = \frac{2e^2 a^2}{3c^3} \linebreak
\end{dmath*}
\begin{dmath*}
    P(\omega) \sim \frac{e^2}{c^3}\int a^2 e^{i\omega t}\d t
              \sim \frac{e^2\delta v^2}{c^3}
              \sim \frac{Z^2e^6}{b^2m_e^2v^2c^3}
\end{dmath*}
\end{dgroup*}
Because the interaction is very rapid and impulsive, the frequency spectrum is \emph{flat}\sidenote{
    If there's one thing you remember about Bremsstrahlung, remember it is flat!
}
out to
a maximum $\omega\sim b/v$ corresponding to the length of the interaction.

To get the spectrum for an ensemble of electrons, we integrate over impact parameters $b$ and
velocities $v$.  The energy radiated per unit time per unit frequency and per unit volume
at a fixed velocity is
\begin{dmath*}
    \frac{\d E}{\d\omega\d V\d t}
        \sim n_en_iv\int_{b_{\rm min}}^{b_{\rm max}} P(\omega) 2\pi b\d b \nolinebreak
        \sim \frac{Z^2e^6}{m_e^2vc^3}n_en_i\ln\left(\frac{b_{\rm max}}{b_{\rm min}}\right)
\end{dmath*},
where $n_e$ and $n_i$ are the number density of electrons and ions respectively.
Using a Maxwellian velocity distribution\sidenote{
    \begin{dmath*}
    P(v)\d v = \left(\frac{m}{2\pi kT}\right)^{3/2}4\pi v^2e^{-mv^2/2kT}\d v
    \end{dmath*}
} and sweeping the details of $b_{\rm max}$ and
$b_{\rm min}$ into a Gaunt factor\sidenote{
    The minimum and maximum impact parameters depend on whether or not quantum mechanics
    is relevant.
}, $g_{ff}$, we have
\begin{dmath*}
    \frac{\d E}{\d\omega\d V\d t}
        \sim \frac{Z^2e^6g_{ff}}{m_e^2c^3}n_en_i\left(\frac{m_e}{kT}\right)^{3/2}\int_{v_{\rm min}}^\infty ve^{-m_ev^2/2kT} \d v
        \sim \frac{Z^2e^6g_{ff}}{m_e^2c^3}n_en_i\left(\frac{m_e}{kT}\right)^{1/2}e^{-m_ev_{\rm min}^2/2kT}
\end{dmath*}.
The minimum velocity $v_{\rm min}$ must be chosen so that a photon of frequency $\nu$ can
be created.  Therefore choose $v_{\rm min}=(2h\nu/m_e)^{1/2}$ and finally we have the emissivity
\begin{dmath}\boxed{
    \epsilon_{ff}
        \sim \frac{e^6}{m_e^{1/2}k^{1/2}c^3}Z^2n_en_iT^{-1/2}e^{-h\nu/kT}g_{ff}
}\end{dmath}.
Note that this spectrum is essentially \emph{flat} until $h\nu\sim kT$.

As an alternative thought\sidenote{
    Michael: I think it's interesting to think about electron's Compton scattering off
    virtual photons conceptually, but I don't think it's elucidating (even slightly).
    I have no intuition about how an electron should Compton scatter virtual photons and
    what the power or spectrum should be.  Thinking about accelerating electrons seems
    more physical and obvious to me -- why complicate the matter?
},
for the Bremsstrahlung power from a single electron, use Thompson scattering in the electron's 
frame.  The electron sees a moving proton, which creates a varying E and B field (ie photons).  
The electron Thompson scatters off these virtual photons, radiating as in Thompson scattering.
Because energy is invariant of frame, the power radiated in this process according to the 
electron is the same as the power we see.  So
\begin{displaymath}P=2\sigma_Tc\langle u_E \rangle=2\sigma_Tc\frac{E^2}{8\pi}=2\sigma_Tc\frac{Z^2e^2}{8\pi b^4}\end{displaymath}
To get the emissivity, need to integrate over possible impact parameters and (Boltzmann) 
velocities of a population of electrons, and consider many possible ions to interact with.  
Know how to describe and explain a Bremsstrahlung spectrum.

\subsection{Cyclotron}
Non-relativistic electrons spiraling in a magnetic field.  From far away, looks like a dipole.  
\begin{displaymath}P=\frac{2e^2a^2}{3c^3}\end{displaymath}
The accelaration can be found from the cyclotron frequency:
\begin{displaymath}a=\dot{v}=\frac{v^2}{r}=v\omega_{cyc}=v\frac{eB}{m_ec}\end{displaymath}
\begin{displaymath}P=\frac{2e^4B^2v^2}{3c^5m_e^2}\end{displaymath}
\begin{displaymath}\boxed{P=\frac{2}{3}r_0^2B^2c\beta^2}\end{displaymath}

\subsection{Synchrotron}
See will ask for derivation of power from single electron.  

To get the spectrum of many electrons, assume a power law distribution of electron energies, and 
multiply this by the power of one electron as a function of energy.  This results in 
\begin{displaymath}j_{\nu}\propto B^{(p+1)/2}\nu^{(1-p)/2}\end{displaymath}
For most synchrotron spectra, $(1-p)/2\equiv \alpha=-0.7$, so
\begin{displaymath}j_{\nu}\propto B^{1.7}\nu^{-0.7}\end{displaymath}
For optically thick synchrotron, absorption becomes important.  In this regime, $h\nu<<kT$, so
\begin{displaymath}I_{\nu}=\frac{2k}{c^2}T\nu^2\end{displaymath}
Since synchrotron is a non-thermal process, T is no longer brightness temperature, and varies 
as a function of frequencies (electrons emitting at different frequencies have different 
temperatures).  Assuming an electron emits at its critical frequency, $\nu\propto\gamma^2$, so 
$\gamma\propto\nu^{1/2}$.  Assuming $kT\sim E=\gamma m_ec^2$, then 
$T\propto\gamma\propto\nu^{1/2}$.  So, the sychrotron spectrum is described by
\begin{displaymath}I_{\nu}\propto \nu^{5/2} (low\ frequency,\ optically\ thick)\end{displaymath}
\begin{displaymath}I_{\nu}\propto \nu^{-0.7} (high\ frequency,\ optically\ thin)\end{displaymath}


\subsection{Synchrotron Self-Absorption}
	I did not even understand what was going on with this when I started studying, so I thought I'd write down what I found out in case this confuses anyone else.
	An astrophysical synchrotron has an ensemble of electrons at various energies, producing emission at their own characteristic frequencies. The energy levels of these electrons are continuous, but can be treated as quantized on the order of $h^{3}$ (not clear about this part really). Emitted photons can change the energy level of another electron by being absorbed and giving up their energy to the electron.
	The amount that this happens depends on the way the electron energies are distributed, both for the emission spectrum and for the distribution of energy levels that can absorb photons of specific energies.
	Result:
	\begin{equation}
	\text{non-thermal electron distribution:} \quad S_{\nu} \propto \nu^{5/2}		
	\end{equation}
	\begin{equation}
	\text{thermal electron distribution:} \quad S_{\nu} \propto \nu^{2}		
	\end{equation}
	
	$\rightarrow$Lower energy photons are more likely to be absorbed, because for them, the synchrotron source is optically thick. You could also say it conversely: the synchrotron source is optically thick for lower energy photons because they are more likely to be absorbed. This is what causes the synchrotron spectrum to turn over and not just continue to infinity for long wavelengths.

Sychrotron Cooling:
Electrons cool by emitting photons, timescale given by
\begin{displaymath}t=\frac{E}{-dE/dt}=\frac{\gamma m_ec^2}{\frac{4}{3}\sigma_Tc\beta^2\gamma^2U_B}\end{displaymath}
\begin{displaymath}\boxed{t_{cool}=\frac{3m_ec}{4\sigma_TU_B\gamma \beta^2}}\end{displaymath}

Equipartition:
Electron and magnetic field energy density both scale with B.
\begin{displaymath}U_e\propto B^{-3/2}\end{displaymath}
\begin{displaymath}U_B\propto B^2\end{displaymath}
At minimum total energy, $U_e\sim U_B$.  Assuming equipartion, the flux from a sychrotron source 
can be used to determine $U_e$, which can then be used to determine $U_B$.

\subsection{Compton Scattering}
In Thomson scattering, collisions are elastic (energy of the photon doesn't change).  In Compton 
scattering, the energy of the photon decreases.  In inverse Compton scattering, a photon gains 
energy from scattering off a high energy electron.  Conserve momentum and energy in the 
scattering to find
\begin{displaymath}\Delta \lambda=\lambda_c(1-\cos{\theta})\end{displaymath}
where $\lambda_c=0.02426$ \AA.
The energy increase for a photon in inverse compton can be derived by treating the scatter as 
regular compton scatter in the frame of the electron, then converting back to the lab frame.  
This gives an energy shift for the photon of 
\begin{displaymath}E_2=\frac{4}{3}E_1\gamma^2\end{displaymath}
For example, in the Crab Nebula, microwave photons from the nebula are inverse compton scattered 
to gamma ray energies.  This is why we see gamma rays from the Crab.  Inverse Compton power 
of a single electron in a photon gas is similar to sychrotron power:
\begin{displaymath}\boxed{P=\frac{4}{3}c\sigma_T\gamma^2\beta^2u_ph}\end{displaymath}
For a population of electrons, $I_{\nu}\propto \nu$ for low $\nu$ and drops off sharply for 
high $\nu$.

\subsection{Sychrotron Self-Compton}
Can get double peaked synchrotron spectra because sychrotron photons are scattered to higher 
energies by all the fast moving electrons around.  This is seen in blazars and the Crab nebula.  
If $P_{IC}>P_{sync}$, the sychrotron photons can be depleted, with none getting out.  Can show 
(as we did in high energy) that this happens for brightness temperatures above $10^{12}$ K.

\subsection{S-Z effect}
Thermal electrons in intra-cluster gas inverse Compton CMB photons to a slightly higher energy 
(this is a non-relativistic effect).  The fractional increase in energy from one collision is 
the power from the that collision divided by the power from all collisions/second.
\begin{displaymath}\frac{\Delta{E}}{E}=\frac{P}{\frac{dN}{dt}E}=\frac{\frac{4}{3}c\sigma_T\beta^2n_{ph}h\nu}{c\sigma_Tn_{ph}}=\frac{4}{3}\beta^2\end{displaymath}
If the population of electrons is thermal, their 3-D rms velocity is $\sqrt{\frac{3kT}{m_e}}$, so
\begin{displaymath}\frac{\Delta{E}}{E}=\frac{4kT}{m_ec^2}\end{displaymath}
For an optically thin gas (which this is), the total number of scatterings is given by the 
optical depth
\begin{displaymath}\tau=\sigma_Tn_e2R\end{displaymath}
So, the fractional shift in the temperature of the CMB blackbody is given by
\begin{displaymath}\frac{\Delta T}{T}=\frac{-2\Delta\nu}{\nu}=\frac{-2\Delta E}{E}\end{displaymath}
can be used to determine the physical radius of the cluster.  Measurements of x-ray thermal 
Bremsstrahlung can give the temperature and density of the electrons, allowing R to be solved for.
Then, the angular size of the cluster can be used to determine its distance.  This, combined with 
its redshift, can be used to measure the Hubble constant.

\subsection{Dispersion}
Light travelling through a plasma is delayed by an amount 
\begin{displaymath}\tau=\int\left(\frac{1}{v_g}-\frac{1}{c}\right)\,ds\end{displaymath}
\begin{displaymath}v_g\sim c\left(1-\frac{\omega_p^2}{\omega^2}\right)^{1/2}\end{displaymath}
So
\begin{displaymath}\tau\sim\frac{1}{2c\omega^2}\int\omega_p^2\,ds=\frac{2\pi e^2}{m_ec}\frac{1}{\omega}DM\end{displaymath}
where $\omega_p^2=\frac{4\pi n_ee^2}{m_e}$ and $DM=\int n_e\,ds$.

\subsection{Hydrogen energy levels}

To calculate the energies of hydrogen levels, the following equation is typically used:

\begin{equation}
E_n = 13.6eV\big(\frac{1}{n_1^2} - \frac{1}{n_2^2}\big)
\end{equation}

Here is the derivation of the Rydberg constant.

\begin{equation}
E_{atom} = \frac{1}{2}mv^2 - \frac{ze^2}{r}
\end{equation}

Balancing forces:

\begin{equation}
\frac{ze^2}{r^2} = \frac{mv^2}{r}
\end{equation}

\begin{equation}
E_{atom} = \frac{-ze^2}{2r}
\end{equation}

Since angular momentum is quantized, we have:

\begin{equation}
L_n = mvr_n = n\big(\frac{h}{2\pi}\big)
\end{equation}

Solving for v and plugging this value back into the force balance equation yields:

\begin{equation}
r_n = \big(\frac{h^2}{4\pi^2me^2}\big)\frac{n^2}{z}
\end{equation}

Atomic energy levels for hydrogen are given by:

\begin{equation}
E_n = \big(\frac{-2\pi^2me^4}{h^2}\big)\frac{1}{n^2}
\end{equation}

\begin{equation}
Ry = \big(\frac{-2\pi^2me^4}{h^2}\big) = -13.6eV
\end{equation}



