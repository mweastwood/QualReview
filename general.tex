\section{General Astronomy}
Some things we come across that aren't specific to any one class, but are 
good to have an idea of.

\begin{table}[H]
\centering
\begin{tabular}{c c}
\hline\hline
Band&$\lambda$\\
\hline
Radio&$>$100 cm\\
Microwave&10 mm-100 cm\\
IR&0.7 $\mu$m-10 mm\\
Optical&400-700 nm\\
UV&10-400 nm\\
Xray&10 pm-10 nm\\
$\gamma$ray&$<$10  pm\\
\hline\hline
\end{tabular}
\caption{Wavelengths of different parts of the Electromagnetic Spectrum}
\end{table}

\begin{table}[H]
\centering
\begin{tabular}{c c c}
\hline\hline
Band&$\lambda$ ($\mu$m)&$\frac{\Delta\lambda}{\lambda}$\\
\hline
U&0.36&0.15\\
B&0.44&0.22\\
V&0.55&0.16\\
R&0.64&0.23\\
I&0.79&0.19\\
J&1.26&0.16\\
H&1.60&0.23\\
K&2.22&0.23\\
\hline\hline
\end{tabular}
\caption{Wavelengths and widths of Johnson system photometric filters}
\end{table}

Good lines to know:
\begin{table}[H]
\centering
\begin{tabular}{c c c}
\hline\hline
Line&Transition&Wavelength (\AA)\\ \hline
Ly$\alpha$&$2\leftrightarrow 1$&1216\AA\\
Ly$\beta$&$3\leftrightarrow 1$&1025\AA\\
Ly$\gamma$&$4\leftrightarrow 1$&972\AA\\
Ly$_{con}$&$\infty\leftrightarrow 1$&911\AA\\
H$\alpha$&$3\leftrightarrow 2$&6563\AA\\
H$\beta$&$4\leftrightarrow 2$&4861\AA\\
H$\gamma$&$5\leftrightarrow 2$&4340\AA\\
H$_{con}$&$\infty\leftrightarrow 2$&3646\AA\\
\hline\hline
\end{tabular}
\caption{Common transitions of Hydrogen.}
\label{tab:spectral_lines}
\end{table}
I think we should know the exact wavelength for Ly$\alpha$,Ly$_{con}$,
H$\alpha$, and H$_{con}$.  Those are all important for other things (Lyman break galaxies, 
H ionization, Balmer jump...).  The other lines, maybe just have some idea of where they are.
Fun fact, the wavelength of Ly$\alpha$ is the address of Cahill!

\subsection{The Virial Theorem}
\newthought{The virial theorem} is a cute result of classical mechanics that can be generally
applied to many areas of astronomy.  Therefore we will treat it here without reference to
stars, galaxies, or the ISM.

Let's begin by thinking about a system of particles labeled with $i=1,\ldots,N$, and with
position and momentum denoted by $\b r_i$ and $\b p_i$ respectively.  By applying the non-relativistic
definition of momentum\sidenote{
    \begin{dmath*}
        \b p_i = m_i\dot{\b r_i}
    \end{dmath*},
    where $m_i$ is the mass of the $i$th particle.
}, we can show
\begin{dmath*}
    \frac{\d}{\d t}\sum_i \b p_i\cdot\b r_i
        = \frac{\d}{\d t}\sum_i m_i\dot{\b r_i}\cdot\b r_i
        = \frac{1}{2}\frac{\d^2}{\d t^2}\sum_i m_i r_i^2
\end{dmath*}.
The final sum is simply the total moment of inertia $I$ of the system\sidenote{
    \begin{dmath*}
        I \equiv \sum_i m_i r_i^2
    \end{dmath*}
}.

However, if we apply the product rule, we can also show
\begin{dmath*}
    \frac{\d}{\d t}\sum_i \b p_i\cdot\b r_i
        = \sum_i \frac{\d\b p_i}{\d t}\cdot\b r_i
            +\sum_i \b p_i\cdot\frac{\d\b r_i}{\d t}
        = \sum_i \b F_i\cdot\b r_i
            +\sum_i m_i{\dot r_i}^2
\end{dmath*}.
The final term we recognize as twice the kinetic energy of the system, $2K$.
Combining this with the previous result gives us the identity\sidenote{
    This is always true if classical mechanics holds!
}
\begin{dmath}
    \frac{1}{2}\frac{\d^2I}{\d t^2}
        = \sum_i \b F_i\cdot\b r_i+2K
\end{dmath}.

Now we specialize to the case of gravitational interactions\sidenote{
    That is, we require
    \begin{dmath*}
        \b F_i = -\sum_{j\neq i} \frac{Gm_im_j}{r_{ij}^3}(\b r_i-\b r_j)
    \end{dmath*}
}.  This implies
\begin{dmath*}
    \sum_i \b F_i\cdot\b r_i
        = -\sum_i\sum_{j\neq i} \frac{Gm_im_j}{r_{ij}^3}(\b r_i-\b r_j)\cdot \b r_i
        = -\sum_i\sum_{j > i} \frac{Gm_im_j}{r_{ij}}
\end{dmath*}.
The final expression is that of the total gravitational potential energy $\Omega$
of the system\sidenote{
    The limit on the second sum is necessary to prevent double-counting.
}.  This gives us the virial theorem
\begin{dmath}\boxed{
    \frac{1}{2}\frac{\d^2I}{\d t^2}
        = \Omega+2K
}\end{dmath}.
A system is said to be virialized if $\d^2I/\d t^2=0$, in which case
\begin{dmath}\boxed{
    \Omega+2K = 0
}\end{dmath}.

Although, we assumed purely gravitational interactions, random thermal collisions
do not contribute to $\sum_i\b F_i\cdot\b r_i$ because the direction of the force is
randomized.  This means we have a very general relation that relates the gravitational potential
energy to the kinetic energy of virialized systems.  Note that virialized systems are
bound because $\Omega+K < 0$.

\subsection{Timescales}
\newthought{The free-fall timescale} describes the length of time an object takes to collapse
under the influence of gravity if its pressure support is removed.  The collapse of a star
during a supernova occurs on the free-fall timescale.
If the Sun had no pressure support and just collapsed under gravity, it's velocity would be 
determined by energy conservation:
\begin{equation}
\frac{1}{2}\left(\frac{dr}{dt}\right)^2 = \frac{GM}{r}-\frac{GM}{r_0}
\end{equation}
assuming the mass interior to any element is constant (the entire cloud collapse all at once).
This equation can be integrated to give
\begin{equation}
\tau_{ff}=\left(\frac{r_0^3}{2GM}\right)^{1/2}\int_0^1\left(\frac{x}{1-x}\right)^{1/2}\,dx
\end{equation}
The definite integral is equal to $\pi/2$, so
\begin{equation}\boxed{
\tau_{ff}=\left(\frac{3\pi}{32G\rho}\right)^{1/2}
}\end{equation}
Free fall time for the Sun is 1800 seconds.

\newthought{The Kelvin-Helmholtz timescale} is the relevant timescale for an object to
radiate away its energy.  Gas clouds collapse into protostars on the Kelvin-Helmholtz timescale,
and stellar cores contract on the Kelvin-Helmholtz timescale when they run out of nuclear fuel.

