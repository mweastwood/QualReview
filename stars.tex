\section{Stars}


\subsection{Questions}
\begin{enumerate}
\item \textbf{Explain what is the Hayashi track, and describe what types of objects live on it.
      Qualitatively explain how it arises and what assumptions are required for its derivation.}
\item \textbf{Estimate the temperature as a function of depth in the Sun's convection zone. What
      is the temperature at the base of the convection zone?}
\item \textbf{Describe the major sources of opacity in stars, and how each depends on density,
      temperature, and metallicity.}
\end{enumerate}


\subsection{Basic Information}

Let's start by listing the properties of the Sun, which are just good to know.

\begin{equation}
L_\odot = 3.8 \times 10^33~{\rm erg}~{\rm s}^-1
\end{equation}

\begin{equation}
R_\odot = 6.9 \times 10^10~{\rm cm}
\end{equation}

\begin{equation}
M_\odot = 2 \times 10^33~{\rm g}
\end{equation}

\begin{equation}
t_\odot = 4.5 \times 10^9~{\rm yrs}
\end{equation}

\begin{equation}
T_{\rm eff} = 5800~{\rm K}
\end{equation}

\begin{equation}
\rho_{\rm c} = 150~{\rm g}~{\rm cm}^{-2}
\end{equation}

\begin{equation}
\langle \rho \rangle_\odot = 150~{\rm g}~{\rm cm}^{-2}
\end{equation}

Ranges for other stars:

\begin{equation}
M \sim 0.1 - 100~{\rm M}_\odot
\end{equation}

\begin{equation}
R \sim 0.1 - 1000~{\rm R}_\odot
\end{equation}

\begin{equation}
L \sim 10^{-4} - 10^6~{\rm L}_\odot
\end{equation}

\begin{equation}
T_{\rm eff} \sim 1000 - 5 \times 10^4~{\rm K}
\end{equation}

Understanding stars: 3 key pieces of physics

1. Force balance: pressure vs. gravity (and sometimes rotation)

2. Energy transport: conduction, radiation, convection

3. Energy generation: fusion, gravitational contraction

Let's talk first about force balance. Assuming no radiation, this is just a competition between pressure and gravity. We can look at a layer in the star with pressure $P$ at its base and $P+dP$ above, and we know the force on the layer due to gravity. The total force will be the pressure difference times the area of the shell $A$ (which will eventually drop out) plus the gravitational force $-\frac{G M_r M_{\rm shell}}{r^2}$. If we assume hydrostatic equilibrium, the total force is zero. We then can switch to differentials to get the hydrostatic equilibrium equation:

\begin{equation}
\frac{dP}{dr} = -\frac{\rho G M_r}{r^2}\,\,.
\end{equation}

...


\subsection{Stellar Evolution}

Collapse from a gas cloud:

The first important thing to know is the Jeans criterion. You can get here by considering a deviation from a system in hydrostatic equilibrium, which is described by the Virial Theorem:

\begin{equation}
2K + U = 0\,\, ,
\end{equation}
where we can approximate the kinetic and gravitational potential energies as

\begin{equation}
U \sim \frac{3 G M_{\rm c}^2}{5 R_{\rm c}}
\end{equation}
\begin{equation}
K = \frac{3}{2} N k_{\rm B} T = \frac{3 M_{\rm c} k_\rm B T}{\mu m_{\rm H}}
\end{equation}

You can assume a constant density for the cloud and find the minimum mass necessary to start collapse, $M_{\rm c} > M_{\rm J}$:

\begin{equation}
M_{\rm J} \approx \biggl( \frac{5 k_{\rm B}T}{G \mu m_{\rm H}}\biggr)^{3/2}\biggl( \frac{3}{4\pi\rho_0}\biggr)^{1/2}
\end{equation}

We can also express this criterion as $R_{\rm c} > R_{\rm J}$, where

\begin{equation}
R_{\rm J} \approx \biggl( \frac{15 k_{\rm B} T}{4 pi G \mu m_{\rm H} \rho_0} \biggr)^{1/2}
\end{equation}

Once the Jeans criterion is satisfied, the cloud basically collapses in freefall at the beginning. It is isothermal, optically thin, and radiates energy efficiently. The freefall time is

\begin{equation}
t_{\rm ff} = \biggl( \frac{3 \pi}{32 G \rho_0}\biggr)^{1/2} \sim \biggl( \frac{1}{G \rho_0}\biggr)^{1/2}
\end{equation}
The numerical factor is from the derivation in Carrol \& Ostlie; you can derive this by considering a particle on the outer edge of and object falling inward due to the gravitational force of all the interior mass.

Also note that fragmentation is an issue: that is, the Jeans mass changes as the density of the cloud goes up, and a natural consequence of this is that a smaller mass is required to collapse. Therefore, one large cloud (of, say, 50 solar masses) might collapse into $\sim 50$ solar-mass clouds.

What eventually stops fragmentation is that the cloud stops being isothermal. The cloud starts transporting more energy adiabatically rather than radiatively because it starts to become optically thick, and radiation does not transport energy out efficiently enough.

(There is some stuff about shocks produced by the outer layers of the cloud free-falling onto the nearly hydrostatic star, which powers it for a while. Not sure how much we need to know about this.)

Pre-Main-Sequence Evolution:

First, the pre-MS protostar collapses on the Kelvin-Helmholtz timescale $t_{\rm KH}$, which for $1~{\rm M}_\odot$ is about $10^7$ years (what is the expression?). The effective temperature increases, so the opacity of the outer layers becomes dominated by the H$^-$ ion, which gets its extra electrons from increased partial ionization of heavy elements. This increased opacity causes the envelope of the protostar to become convective and sometimes this envelope can become very large, reaching even to the center of the star. This object evolves along the Hayashi track, its luminosity decreasing while the effective temperature increases only slightly (it is almost constant, $\sim 3800$ K).

\subsection{Atmospheres}
\subsection{Interiors}






