\section{Stars}

FYI- I looked through the old qual questions we got, and it seems like almost everyone is asked 
something about the HR Diagram, so we should know that inside and out.  Also, questions about 
spectral lines are more popular than I thought they would be.  Globular clusters were also popular.

\subsection{Questions}
\begin{enumerate}
\item \textbf{Explain what is the Hayashi track, and describe what types of objects live on it.
      Qualitatively explain how it arises and what assumptions are required for its derivation.}
\item \textbf{Estimate the temperature as a function of depth in the Sun's convection zone. What is the temperature at the base of the convection zone?}

	%Imagine a blob of gas inside a gaseous medium whose density deviates from the density of its environment. If the blob's density is smaller than the surrounding density, the system is unstable to convection (we should also know how to express this in terms of entropy).
	%Imagine that the blob has $\rho_{b1},~P_{b1},~T_{b1}$, and the surrounding medium has $\rho_1,~P_1,~T_1$. The blob moves up a distance $\delta r$, where the environment has $\rho_2,~P_2,~T_2$, and now the blob has $\rho_{b2},~P_{b2},~T_{b2}$. 
	Assume the process is adiabatic, so the blob obeys $P = K \rho ^\gamma$, and assume we have an ideal gas, so $P = \frac{\rho k_{\rm B} T}{\mu m_p}$. First differentiate the ideal gas law to obtain
	
	\begin{equation}
	\frac{dP}{dr} = \frac{P}{\rho} \frac{d \rho}{dr} + \frac{P}{T} \frac{d T}{dr} \,\, ,
	\end{equation}
	assuming that the pressure does not vary much with the mean molecular weight $\mu$. We can also differentiate the adiabatic pressure equation
	\begin{equation}
	\frac{dP}{dr} = \gamma \frac{P}{\rho}\frac{d\rho}{dr}\,\, .
	\end{equation}
	Setting these two equal, we get
	\begin{equation}
	\biggl(\frac{dT}{dr}\biggr)_{\rm ad} = \biggl(1 - \frac{1}{\gamma}\biggr)\frac{T}{P} \frac{dP}{dr}\,\, ,
	\end{equation}
	which is the adiabatic temperature gradient. In a star, we can also use the ideal gas law to replace $P$ and hydrostatic equilibrium to replace $dP/dr$:
	\begin{equation}
	\biggl(\frac{dT}{dr}\biggr)_{\rm ad} = -\biggl(1 - \frac{1}{\gamma}\biggr)\frac{\mu m_p}{k_{\rm B}} \frac{G M_r}{r^2}\,\, .
	\end{equation}
	This describes how the temperature in the bubble changes as it rises and expands adiabatically. If $dT/dr$ is steeper than the adiabatic temperature gradient, the fluid is superadiabatic and there will be convection.
	
	The convection zone of the Sun extends about $1/3 {\rm R}_\odot$ down from the surface. Assume $M_r$ is constant and is the mass of the Sun, i.e. relatively little mass is contained in the convection zone. For an ideal monatomic gas, $\gamma = 5/3$, and let's assume $\mu \approx 1/2$, since in the outer layers of the Sun it should be mostly hydrogen. We can integrate our equation to get $T(R)$:
	
	\begin{equation}
	\int^{T(R)}_{T_{\rm eff}} dT = -\frac{m_p}{5 k_{\rm B}} G {\rm M}_\odot \int^R_{\rm R_\odot} r^{-2}dr \,\, .
	\end{equation}
	Alternatively, you could integrate from the base of the convection zone if you know what the temperature is there ($1.8 \times 10^6~K$) and can remember that it's about a third of the way in. The final formula is:
	\begin{equation}
	T(R) = \frac{m_p}{5 k_{\rm B}} G {\rm M}_\odot \biggl( \frac{1}{R} - \frac{1}{{\rm R}_\odot}\biggr) + 5800~{\rm K}\,\, .
	\end{equation}
	And you can plug numbers in yourself (someone check the form of this equation and make sure io got it right).
	
\item \textbf{Describe the major sources of opacity in stars, and how each depends on density,
      temperature, and metallicity.}
      
      Electron scattering (Thomson): This opacity is due to photons scattering off free electrons. This opacity is generally close to constant, since the Thomson cross-section is a constant given by
      \begin{equation}
      \sigma_{\rm T} = \frac{8 \pi e^4}{3 c^4 m_e^2} = 6.6524\times 10^{-25}~{\rm cm}^2\,\, .
      \end{equation}
      The opacity $\kappa = \sigma / m$, so the opacity will depend on the total mass per scattering electron; that is, it will depend on the composition of the material and also how much it is ionized. Because of the ionization state, it will also depend on the temperature, i.e. whether the material is fully ionized. For a fully ionized medium, however, the opacity should be roughly constant, since there will typically be 1 or 2 baryons per electron, so $m \sim 1-2~m_p$. Therefore $\kappa$ is usually of the order 0.1-0.4 cm$^2$ g$^{-1}$.

      The exact opacity (assuming a fully ionized gas) from Thomson scattering can be calculated 
      as a function of H mass fraction.  In a similar calculation to the mean molecular weight 
      derivation, 
      \begin{equation}
      n_e=n_H+2n_{He}+\Sigma{A}{2}n_A=\frac{\rho}{m_H}\left(X+\frac{2}{4}Y+\frac{1}{2}Z\right)=\frac{\rho}{2m_H}\left(1+X\right)
      \end{equation}
      So,
      \begin{equation}
      \kappa_T=\frac{n_e\sigma_T}{\rho}=\frac{sigma_T}{2m_H}\left(1+X\right)=\left(1+X\right)0.2~{\rm cm^2 g^-1}
      \end{equation}
       
      H$^-$ ion: Occasionally, if there are enough free electrons present (easier with increased partial ionization of heavy elements), some will become weakly bound to hydrogen atoms, so that the proton is surrounded by two electrons. Opacity is caused by photoionization of this extra electron with binding energy of 0.75 eV. 
      
      \begin{equation}
      \kappa_{H^{-}} \propto Z \rho^{1/2} T^{9}
      \end{equation}
      
      The opacity is very sensitive to temperature, because a fragile balancing act is required. The temperature must be less than 10,000K (when Hydrogen ionizes) so that neutral Hydrogen is around to get another electron. But, the temperature must also be high enough so that an abundance of free electrons are around which were liberated from ionized heavy elements. This opacity is also sensitive to metallicity, for the same reason. 
      Despite $H^{-}$ being a fragile and rare ion, this type of opacity is the most important source in our Sun's atmosphere, and is dominant in FGK stars.

      Averaging bound-free and free-free absorption over all wavelenthts gives Kramer's opacity:
      \begin{equation}
      \bar{\kappa}\propto \frac{\rho}{T^{3.5}}
      \end{equation}
       
\end{enumerate}

\subsection{Stellar Properties}
Let's start by listing the properties of the Sun, which are just good to know.
\begin{dgroup}
\begin{dmath}
L_\astrosun = 3.8 \times 10^{33}~{\rm erg}~{\rm s}^-1
\end{dmath}
\begin{dmath}
R_\astrosun = 6.9 \times 10^{10}~{\rm cm}
\end{dmath}
\begin{dmath}
M_\astrosun = 2 \times 10^{33}~{\rm g}
\end{dmath}
\begin{dmath}
t_\astrosun = 4.5 \times 10^9~{\rm yrs}
\end{dmath}
\begin{dmath}
T_{\rm eff} = 5800~{\rm K}
\end{dmath}
\begin{dmath}
T_{\rm c} = 15 \times 10^6~{\rm K}
\end{dmath}
\begin{dmath}
\rho_{\rm c} = 150~{\rm g}~{\rm cm}^{-3}
\end{dmath}
\begin{dmath}
\langle \rho \rangle_\astrosun = 1~{\rm g}~{\rm cm}^{-3}
\end{dmath}
\end{dgroup}
Ranges for other stars:
\begin{dgroup}
\begin{dmath}
M \sim 0.1 - 100~{\rm M}_\astrosun
\end{dmath}
\begin{dmath}
R \sim 0.1 - 1000~{\rm R}_\astrosun
\end{dmath}
\begin{dmath}
L \sim 10^{-4} - 10^6~{\rm L}_\astrosun
\end{dmath}
\begin{dmath}
T_{\rm eff} \sim 1000 - 5 \times 10^4~{\rm K}
\end{dmath}
\end{dgroup}

\subsection{Spectral Types}
O, B, A, F, G, K, M\newline
Within these types, stars are numbered 0 through 9 (hottest to coolest).  Luminosity classes 
are also used.  I-IV are giant stars, while V are main sequence stars.  D is for white dwarfs.  
So, for example, the Sun is a G2V star.\newline
O stars:\newline
>16 $M_{\odot}$\newline
Very weak spectral lines (just about everything is ionized, so there's nothing 
to produce transitions).  Very weak Balmer lines and some weak lines to due 
ionization of He.  No Balmer break visible.\newline
B stars:\newline
2-16 $M_{\odot}$\newline
Stronger Balmer lines, also He absorption lines.  Still a relatively clean 
spectrum though.  Balmer break visible (and visible in all later spectral 
types.\newline
A stars:\newline
1.5-2 $M_{\odot}$\newline
Strongest Balmer lines because these are the hottest stars (hot enough to have 
lots of H in the n=2 state) without being too hot and ionizing all the H.  
Effective temperature of $\sim$10,000 K is perfect for Balmer lines.\newline
F stars:\newline
1-1.5 $M_{\odot}$\newline
Balmer lines start to weaken again.  Star seeing metal lines (mainly neutral 
or singly ionized Ca, Mg, and Na).\newline
G stars:\newline
0.8-1 $M_{\odot}$\newline
This includes the Sun.  Weaker Balmer lines, stronger metal lines.\newline
K stars:\newline
0.5-0.8 $M_{\odot}$\newline
Weaker Balmer, stronger metal.  Also start seeing absorption bands from 
molecules (smearing of lots of absorption lines due to rotational, vibrational, 
and electronic transitions).
M stars:\newline
<0.5 $M_{\odot}$\newline
Dominated by molecular bands, in particular TiO between 6600 \AA\ and 8600 \AA.

\subsection{HR Diagram}
Be able to draw and explain one.  Some useful approximate relations for the main sequence:
\begin{equation}L\propto T_{eff}^8\end{equation}
\begin{equation}R\propto T_{eff}^2\end{equation}
\begin{equation}L\propto M^3\ for\ stars\ more\ massive\ than\ the\ Sun\end{equation}
\begin{equation}L\propto M^5\ for\ stars\ less\ massive\ than\ the\ Sun\end{equation}
Main sequence masses range from $\sim 0.1\ M_{\odot}$ to $\sim 100\ M_{\odot}$.  Main sequence 
radii range from $\sim 0.25\ R_{\odot}$ to $\sim 25\ R_{\odot}$.

\subsection{Timescales}
The free-fall timescale and Kelvin-Helmholtz timescale are the most relevant here.  We should
include a definition and derivation
Free-fall Time:\newline
If the Sun had no pressure support and just collapsed under gravity, it's velocity would be 
determined by energy conservation:
\begin{equation}
\frac{1}{2}\left(\frac{dr}{dt}\right)^2 = \frac{GM}{r}-\frac{GM}{r_0}
\end{equation}
assuming the mass interior to any element is constant (the entire cloud collapse all at once).
This equation can be integrated to give
\begin{equation}
\tau_{ff}=\left(\frac{r_0^3}{2GM}\right)^{1/2}\int_0^1\left(\frac{x}{1-x}\right)^{1/2}\,dx
\end{equation}
The definite integral is equal to $\pi/2$, so
\begin{equation}
\tau_{ff}=\left(\frac{3\pi}{32G\rho}\right)^{1/2}
\end{equation}
Free fall time for the Sun is 1800s.

\subsection{The Virial Theorem}
I will add a nice description of this when I'm not tired -- Michael

Here's a description from Astrophysics in a nutshell.  It's pretty conceptual, so feel free 
to add to it Michael.
Start with the hydrostatic equilibrium equation, multiply both sides by $4\pi r^3dr$, and integrate:
\begin{equation}
\int_0^R4\pi r^3\frac{dP}{dr}\,dr = -\int_0^R\frac{GM(r)\rho (r)4\pi r^2}{r}\,dr
\end{equation}
The right hand side is just the total gravitational energy in the star.  The left hand side 
can be integrated by parts to give
\begin{equation}
[P(r)4\pi r^3]_0^R-3\int_0^RP(r)4\pi r^2\,dr
\end{equation}
The first term is 0 because pressure goes to 0 at the surface.  The second term is average 
pressure times volume, so 
\begin{equation}
\bar{P}=-\frac{1}{3}\frac{E_{gr}}{V}
\end{equation}
Using the ideal gas law and the thermal energy of a gas ($\frac{3}{2}NkT$), can show
\begin{equation}
P=\frac{2}{3}\frac{E_{th}}{V}
\end{equation}
So, 
\begin{equation}
\boxed{E_{th}=-\frac{E_{gr}}{2}}
\end{equation}
which is the Virial theorem.  This is why stars have a negative heat capacity and evolve the way 
they do.  If they contract (making $E_{gr}$ more negative), the thermal energy (temperature) 
increases.

\subsection{Equations of Stellar Structure}
\newthought{There are three} key pieces of physics that are necessary for understanding stars:
\begin{enumerate}
    \item Force balance: pressure vs. gravity (and sometimes rotation)
    \item Energy transport: conduction, radiation, convection
    \item Energy generation: fusion, gravitational contraction
\end{enumerate}
These form the essence of the equations of stellar structure.
\begin{dgroup*}
\begin{dmath}
    \frac{\d M_r}{\d r} = 4\pi r^2\rho \quad \text{(mass continuity)}
\end{dmath}
\begin{dmath}
    \frac{\d P}{\d r} = -\frac{GM_r\rho}{r^2} \quad \text{(hydrostatic equilibrium)}
\end{dmath}
\end{dgroup*}

Let's talk first about force balance. Assuming no radiation, this is just a competition between pressure and gravity. We can look at a layer in the star with pressure $P$ at its base and $P+dP$ above, and we know the force on the layer due to gravity. The total force will be the pressure difference times the area of the shell $A$ (which will eventually drop out) plus the gravitational force $-\frac{G M_r M_{\rm shell}}{r^2}$. If we assume hydrostatic equilibrium, the total force is zero. We then can switch to differentials to get the hydrostatic equilibrium equation:

\begin{equation}
\frac{dP}{dr} = -\frac{\rho G M_r}{r^2}\,\,.
\end{equation}

The mass continuity equatin just comes from the amount of mass in a spherical shell at radius r 
and of thickness dr.

The radiative temperature gradient can be derived by considering the energy in shells at different 
radii.  Since energy flows out of the star, the energy density must decrease outwards.  Consider 
a shell with energy denstity $u+du$ (shell 1), with another shell $dr$ above it of energy 
density $u$ (shell 2).  The energy flow from shell 1 to shell 2 per unit time ($L(r)$) is 
the excess energy in shell 1 divided by the time it takes for this energy to flow from shell 1 to 
shell 2.  So
\begin{equation}\label{eq:L}
L(r)=-\frac{4\pi r^2\,dr\,du}{(dr)^2/lc}
\end{equation}
where the denominator is the time it takes a the photon to cross from shell 1 to shell 2 by random 
walk.  Taking the angle of the radiation into account introduces a factor of 1/3 on the right hand 
side.  Equation \ref{eq:L} can be rewritten as 
\begin{equation}
\frac{L(r)}{4\pi r^2}=-\frac{cl}{3}\frac{du}{dr}
\end{equation}
This can also be thought of as a diffusoin equation, where the left hand side is energy flux, 
$\frac{cl}{3}$ is the diffusion coefficient, and $\frac{du}{dr}$ is the energy density gradient.  
Plugging in $u=aT^4$ and $l=\frac{1}{\kappa\rho}$, and solving for $\frac{dT}{dr}$ gives
\begin{equation}
\boxed{\frac{dT(r)}{dr}=-\frac{3L(r)\kappa(r)\rho(r)}{4\pi r^2 4acT^3(r)}}
\end{equation}

The energy conservation equation just comes from defining $\epsilon(r)$ as the power generated 
per unit mass.  With this definition,
\begin{equation}
\boxed{\frac{dL(r)}{dr}=4\pi r^2\rho (r)\epsilon (r)}
\end{equation}

In addition to the four main equations of stellar structure, three more equations must be 
specified for the pressure, opacity, and energy generation.  
\begin{equation}
P=P(\rho,T,composition)
\end{equation}
\begin{equation}
\kappa=\kappa(\rho,T,composition)
\end{equation}
\begin{equation}
\epsilon=\epsilon(\rho,T,composition)
\end{equation}
Also, the four differential equations require four boundary conditions:
\begin{equation}
M(r=0)=0
\end{equation}
\begin{equation}
L(r=0)=0
\end{equation}
\begin{equation}
P(r=R)=0
\end{equation}
\begin{equation}
M(r=R)=M_*
\end{equation}

So, there are 7 equations for 7 unknowns, and 4 boundary conditions for 4 differential equations.  
This means that there is a unique solution given the composition and total mass of the star, the 
only two free parameters in the equations.  This is the \emph{Vogt-Russell conjecture}, that the 
properties and evolution of a star depend only on its mass and initial composition.

The equation of state is almost always ideal gas, but here's a good derivation of mean molecular 
weight, which I personally can never remember.

For a fully ionized gas, H contributes 2 particles, He contributes 3, and metals contribue 
$\sim\frac{A}{2}$.  So the total number density of particles is
\begin{equation}
n=2n_H+3n_{He}+\Sigma \frac{A}{2}n_A=\frac{rho}{m_H}\left(2X+\frac{3}{4}Y+\frac{1}{2}Z\right)=\frac{\rho}{2m_H}\left(3X+\frac{1}{2}Y+1\right)
\end{equation}
using the fact that $X+Y+Z=1$.  Then
\begin{equation}
\mu=\frac{\rho}{nm_H}=\frac{2}{1+3X+0.5Y}
\end{equation}

io likes the following definition of mean molecular weight:
\begin{equation}
\frac{1}{\mu_I} = \sum \frac{X_i}{A_i}\,\, ,
\end{equation}
\begin{equation}
\frac{1}{\mu_e} = \sum \frac{Z_iX_i}{A_i} \,\, ,
\end{equation}
\begin{equation}
\frac{1}{\mu} = \biggl(\frac{1}{\mu_I} + \frac{1}{\mu_e} \biggr)^{-1}\,\,.
\end{equation}

See the will ask question for a discussion of opacity.

Also, radiation pressure can contribute to the overall pressure (and sometimes dominate).  
For a photon gas,
\begin{equation}
P_{rad}=\frac{1}{3}u=\frac{1}{3}aT^4
\end{equation}




\subsection{Stellar Evolution}

Collapse from a gas cloud:

The first important thing to know is the Jeans criterion. You can get here by considering a deviation from a system in hydrostatic equilibrium, which is described by the Virial Theorem:

\begin{equation}
2K + U = 0\,\, ,
\end{equation}
where we can approximate the kinetic and gravitational potential energies as

\begin{equation}
U \sim \frac{3 G M_{\rm c}^2}{5 R_{\rm c}}
\end{equation}
\begin{equation}
K = \frac{3}{2} N k_{\rm B} T = \frac{3 M_{\rm c} k_\rm B T}{\mu m_{\rm H}}
\end{equation}

You can assume a constant density for the cloud and find the minimum mass necessary to start collapse, $M_{\rm c} > M_{\rm J}$:

\begin{equation}
M_{\rm J} \approx \biggl( \frac{5 k_{\rm B}T}{G \mu m_{\rm H}}\biggr)^{3/2}\biggl( \frac{3}{4\pi\rho_0}\biggr)^{1/2}
\end{equation}

We can also express this criterion as $R_{\rm c} > R_{\rm J}$, where

\begin{equation}
R_{\rm J} \approx \biggl( \frac{15 k_{\rm B} T}{4 pi G \mu m_{\rm H} \rho_0} \biggr)^{1/2}
\end{equation}

Once the Jeans criterion is satisfied, the cloud basically collapses in freefall at the beginning. It is isothermal, optically thin, and radiates energy efficiently. The freefall time is

\begin{equation}
t_{\rm ff} = \biggl( \frac{3 \pi}{32 G \rho_0}\biggr)^{1/2} \sim \biggl( \frac{1}{G \rho_0}\biggr)^{1/2}
\end{equation}
The numerical factor is from the derivation in Carrol \& Ostlie; you can derive this by considering a particle on the outer edge of and object falling inward due to the gravitational force of all the interior mass.

Also note that fragmentation is an issue: that is, the Jeans mass changes as the density of the cloud goes up, and a natural consequence of this is that a smaller mass is required to collapse. Therefore, one large cloud (of, say, 50 solar masses) might collapse into $\sim 50$ solar-mass clouds.

What eventually stops fragmentation is that the cloud stops being isothermal. The cloud starts transporting more energy adiabatically rather than radiatively because it starts to become optically thick, and radiation does not transport energy out efficiently enough.

(There is some stuff about shocks produced by the outer layers of the cloud free-falling onto the nearly hydrostatic star, which powers it for a while. Not sure how much we need to know about this.)

Pre-Main-Sequence Evolution:

First, the pre-MS protostar collapses on the Kelvin-Helmholtz timescale $t_{\rm KH}$, which for $1~{\rm M}_\astrosun$ is about $10^7$ years (what is the expression?). The effective temperature increases, so the opacity of the outer layers becomes dominated by the H$^-$ ion, which gets its extra electrons from increased partial ionization of heavy elements. This increased opacity causes the envelope of the protostar to become convective and sometimes this envelope can become very large, reaching even to the center of the star. This object evolves along the Hayashi track, its luminosity decreasing while the effective temperature increases only slightly (it is almost constant, $\sim 3800$ K).


\subsection{Radial Velocities}

Derivation of radial velocity amplitude from a star and a low mass companion (say a planet).

Let $r_p$ be the distance between the planet and the center of mass of the system, and $r_s$ be the distance between the star and the center of mass.

\begin{equation}
a = r_p + r_s
\end{equation}

\begin{equation}
v_p = \frac{2\pi r_p}{P}
\end{equation}
\begin{equation}
v_s = \frac{2\pi r_s}{P}
\end{equation}

\begin{equation}
M_pr_p = M_sr_s
\end{equation}

Combining above equations yields the following relation:
\begin{equation}
\frac{v_p}{v_s} = \frac{r_p}{r_s} = \frac{M_s}{M_p}
\end{equation} 

\begin{equation}
a = r_p + r_s = (\frac{M_s}{M_p} + 1)r_s = r_s\big(\frac{M_s + M_p}{M_s}\big)
\end{equation}

Plugging the above expression into Kepler's 3rd law:

\begin{equation}
P^2 = \frac{4\pi^2r_s^3(M_p + M_s)^2}{GM_p^3}
\end{equation}

Plugging in for $r_s$ and rearranging a bit:

\begin{equation}
v_s^3 = \frac{P^{-1}2\pi GM_p^3}{(M_p + M_s)}
\end{equation}

By assuming the companion is significantly less massive than the star, you can simplify the expression for the radial velocity amplitude K.  
\begin{equation}
v_s = K = 2\pi G P^{-\frac{1}{3}}M_pM_s^{-\frac{2}{3}}
\end{equation}

\subsection{Atmospheres}
\subsection{Interiors}






