\section{Galaxies}
\subsection{Questions}
galaxies suck!!!!
\begin{enumerate}
\item \textbf{Define two-body relaxation and estimate its time scale in (a) a globular cluster; (b)
      the Milky Way's disk. In which cases is two-body relaxation important?}
      
      Here's one way of calculating two-body relaxation. Assume $N$ bodies of mass $m$. Look at one mass travelling past another mass with relative velocity $v$ and impact parameter $b$. Assume that the time for the two to interact gravitationally is $2b/v$, that is, they are only interacting while the moving mass is $b$ away from its point of closest approach. The acceleration the mass feels due to the other mass is on the order of $Gm/b^2$. Therefore the change in velocity is
      \begin{equation}
      \delta v_\perp (v,b) \sim \frac{Gm2b}{b^2v} = \frac{2Gm}{bv}\,\,.
      \end{equation}
      When looking at many interactions over time, $(\delta v_\perp)^2$ grows with time.
      \begin{equation}
      \frac{d}{dt}(\delta \bar{v_\perp})^2 \sim \int^{b_{\rm max}}_{b_{\rm min}}\delta v_\perp^2(v,b) n v 2 \pi b db \sim \frac{8 \pi G^2 m^2 n}{v} \ln \biggl( \frac{b_{\rm min}}{b_{\rm max}} \biggr) \,\, .
      \end{equation}
      The factor of $n v 2 \pi b db$ comes from thinking about many masses streaming past our central mass. The flux of masses in number per second per area is $nv$. Then we can imagine a ring of radius $b$ through which the particles are streaming. It has circumference $2 \pi b$ and thickness $db$, so the area of this thin ring can be multiplied by the flux to get the number of interactions per time. Then we just multiply the rate of interactions by the squared change in velocity. We can approximate the relaxation time as
      \begin{equation}
      t_R = \frac{v^2}{\frac{d}{dt}(\delta \bar{v_\perp})^2} = \frac{v^3}{8 \pi G^2 m^3 n \ln(b_{\rm max}/b_{\rm min})} \,\,.
      \end{equation}
\item \textbf{Draw qualitatively the spectral energy distribution of the Milky Way, and describe
      how its morphology might appear to an external observer as a function of wavelength.}
\item \textbf{Describe at least three methods to probe the gravitational potential of galaxies,
      their assumptions, and their realm of applicability.}
\end{enumerate}

\subsection{``The Galaxy''}

Let's describe the strucutre of the Milky Way here.  I'll begin with the thick vs thin disk
because that's interesting to me.

\newthought{In 1983, Gilmore and Reid} noted two separate structures within the disk of the
Milky Way: the thin disk ($\sim1$ kpc scale height) and the thick disk
($\sim5$ kpc scale height).  The distinction (apart from the obvious difference in dynamics)
is that the thick disk consists of an older population of stars while the thin disk consists
of a younger population of stars.
The stars in the thick disk are believed to be have been formed in a thinner disk and then
excited to larger scale heights through encounters with satellite galaxies or mergers.
The thin disk is believed to have been formed by gas
accretion at later times during the formation of the galaxy.

I believe we had a colloquium earlier this year (or last year) that used SDSS data to show
that there is no thin disk/thick disk distinction but rather a continuous distribution
where the scale height of a stellar population scales with its age.

