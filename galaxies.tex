\section{Galaxies}
\subsection{Questions}
galaxies suck!!!!
\begin{enumerate}
\item Define two-body relaxation and estimate its time scale in (a) a globular cluster; (b)
      the Milky Way's disk. In which cases is two-body relaxation important?
\item Draw qualitatively the spectral energy distribution of the Milky Way, and describe
      how its morphology might appear to an external observer as a function of wavelength.
\item Describe at least three methods to probe the gravitational potential of galaxies,
      their assumptions, and their realm of applicability.
\end{enumerate}

\subsection{``The Galaxy''}

Let's describe the strucutre of the Milky Way here.  I'll begin with the thick vs thin disk
because that's interesting to me.

\newthought{In 1983, Gilmore and Reid} noted two separate structures within the disk of the
Milky Way: the thin disk ($\sim1$ kpc scale height) and the thick disk
($\sim5$ kpc scale height).  The distinction (apart from the obvious difference in dynamics)
is that the thick disk consists of an older population of stars while the thin disk consists
of a younger population of stars.
The stars in the thick disk are believed to be have been formed in a thinner disk and then
excited to larger scale heights through encounters with satellite galaxies or mergers.
The thin disk is believed to have been formed by gas
accretion at later times during the formation of the galaxy.

I believe we had a colloquium earlier this year (or last year) that used SDSS data to show
that there is no thin disk/thick disk distinction but rather a continuous distribution
where the scale height of a stellar population scales with its age.

