\section{Galaxies}
\subsection{Questions}
galaxies suck!!!!
\begin{enumerate}
\item \textbf{Define two-body relaxation and estimate its time scale in (a) a globular cluster; (b)
      the Milky Way's disk. In which cases is two-body relaxation important?}
      
      Here's one way of calculating two-body relaxation. Assume $N$ bodies of mass $m$. Look at one mass travelling past another mass with relative velocity $v$ and impact parameter $b$. Assume that the time for the two to interact gravitationally is $2b/v$, that is, they are only interacting while the moving mass is $b$ away from its point of closest approach. The acceleration the mass feels due to the other mass is on the order of $Gm/b^2$. Therefore the change in velocity is
      \begin{equation}
      \delta v_\perp (v,b) \sim \frac{Gm2b}{b^2v} = \frac{2Gm}{bv}\,\,.
      \end{equation}
      When looking at many interactions over time, $(\delta v_\perp)^2$ grows with time.
      \begin{equation}
      \frac{d}{dt}(\delta \bar{v_\perp})^2 \sim \int^{b_{\rm max}}_{b_{\rm min}}\delta v_\perp^2(v,b) n v 2 \pi b db \sim \frac{8 \pi G^2 m^2 n}{v} \ln \biggl( \frac{b_{\rm max}}{b_{\rm min}} \biggr) \,\, .
      \end{equation}
      The factor of $n v 2 \pi b db$ comes from thinking about many masses streaming past our central mass. The flux of masses in number per second per area is $nv$. Then we can imagine a ring of radius $b$ through which the particles are streaming. It has circumference $2 \pi b$ and thickness $db$, so the area of this thin ring can be multiplied by the flux to get the number of interactions per time. Then we just multiply the rate of interactions by the squared change in velocity. We can approximate the relaxation time as
      \begin{equation}
      t_R = \frac{v^2}{\frac{d}{dt}(\delta \bar{v_\perp})^2} = \frac{v^3}{8 \pi G^2 m^2 n \ln(b_{\rm max}/b_{\rm min})} \,\,.
      \end{equation}
      Let's say $b_{\rm max} \sim R \sim (N/n)^{1/3}$, where $N$ is the total number of bodies in the system, while $b_{\rm max} \sim n^{-1/3}$. Therefore
      \begin{equation}
      \frac{b_{\rm max}}{b_{\rm min}} \sim N^{1/3}\,\,.
      \end{equation}
      
      The average density of stars in a globular cluster is $0.4~{\rm pc}^{-3}$, and we can probably assume the average mass is something like $1-2~{\rm M}_\odot$.
\item \textbf{Draw qualitatively the spectral energy distribution of the Milky Way, and describe
      how its morphology might appear to an external observer as a function of wavelength.}
\item \textbf{Describe at least three methods to probe the gravitational potential of galaxies,
      their assumptions, and their realm of applicability.}
      
      Radial velocity curves of spiral galaxies: using the velocities measured by spectral lines at various radii in the galacy. Can be applied to spiral galaxies but not ellipticals or irregulars (I think)\sidenote{
        Michael -- I'm not sure this is true.  I think every galaxy has some rotational component
        and some velocity dispersion component.  Ellipticals have some coherent rotation as well.
        I just can't recall ever seeing a rotational velocity as a function of radius for a
        non-spiral galaxy.
      }. Here you're assuming Keplerian motion of the stars in the disk and that you can easily obtain disk velocities from the radial velocity by knowing how the galaxy's plane is projected on the sky.
      
      Velocity dispersions of ellipticals: here you can use the width of spectral lines to find the velocity distribution of stars. Limited to ellipticals (and maybe bulges of spirals?).
      
      Gravitational lensing: use the angle that light is gravitationally bent (corresponding to the Einstein radius) to measure the mass of a galaxy or galaxy cluster. Usually it's harder to get detailed information about the potential in space, since you'll normally just have one or a few lensed objects behind your foreground object. You also need to pick a foreground objects with lensable objects behind it.
      
\end{enumerate}

\subsection{``The Galaxy''}

Let's describe the strucutre of the Milky Way here.  I'll begin with the thick vs thin disk
because that's interesting to me.

\newthought{In 1983, Gilmore and Reid} noted two separate structures within the disk of the
Milky Way: the thin disk ($\sim1$ kpc scale height) and the thick disk
($\sim5$ kpc scale height).  The distinction (apart from the obvious difference in dynamics)
is that the thick disk consists of an older population of stars while the thin disk consists
of a younger population of stars.
The stars in the thick disk are believed to be have been formed in a thinner disk and then
excited to larger scale heights through encounters with satellite galaxies or mergers.
The thin disk is believed to have been formed by gas
accretion at later times during the formation of the galaxy.

I believe we had a colloquium earlier this year (or last year) that used SDSS data to show
that there is no thin disk/thick disk distinction but rather a continuous distribution
where the scale height of a stellar population scales with its age.

\subsection{The Luminosity Function}

The luminosity function $\Phi(M)dM$ is the number of objects per volume in an absolute magnitude range $M$ to $M+dM$. 

In the context of stars, note that while faint K and M dwarfs make up most of the stellar mass density in the solar neighborhood, almost all emitted light comes from the rare, luminous stars. The peak of the stellar luminosity function occurs at about $M_v = 12.5$, as a result of the rapid luminosity decline of main sequence stars as their mass approaches the limit of hydrogen burning ( $M_{crit} = 0.08 M_\odot$)

If the stars in a given population were born at different times, the stellar luminosity function must be corrected for the effects of stellar evolution. If the star formation rate is constant, as it is in the solar neighborhood,  the corrected luminosity function is given by:

 \begin{equation}
\Phi_0(M) = \Phi(M) \times
\begin{cases}
\frac{t}{\tau_{MS}(M)} & \text{for } \tau_{MS}(M)<t \\
1 & \text{otherwise}
\end{cases}
\end{equation} 
where $t$ is the time since formation started and $\tau_{MS}(M)$ is the main sequence lifetime of stars with absolute magnitude $M$. Note that we only observe stars of magnitude $M$ that were formed in the last $\frac{t}{\tau_{MS}(M)}$ fraction of the population's lifetime.

In the context of galaxies, the Schechter luminosity function provides a general analytic fit to galaxy functions. The Schechter function can be expressed in terms of luminosity:

\begin{equation}
\Phi(L) dL \propto L^\alpha  e^{\frac{-L}{L_*}} dL
\end{equation}
or in terms of absolute magnitude:

\begin{equation}
\Phi(M) dM \propto 10^{-0.4(\alpha+1)M} e^{-10^{0.4(M^*-M)}} dM.
 \end{equation}
 The parameters $\alpha$, $L^*$, and $M^*$ are usually fit to a given dataset. For galaxies near the Milky Way, $\alpha = -1$ and $M^*_B = -21$. For the Milky Way itself, $L_* = 3\times 10^{10} L_\odot$. In the Schechter function, $M^*$ and $L^*$ are also the characteristic values at which the number of galaxies falls off sharply. x 

\section{The Initial Mass Function}

After a burst of star formation, the number of stars in a mass range $\mathcal{M}$ to $\mathcal{M}+d\mathcal{M}$ is given by:

\begin{equation}
dN = N_0 \xi(\mathcal{M}) d\mathcal{M}
\end{equation}
where $\xi(\mathcal{M})$ is the number of stars of mass $\mathcal{M}$, and $N_0$ depends on the normalization of $\xi$. The IMF is normalized such that:

\begin{equation}
\int d\mathcal{M} \mathcal{M} \xi(\mathcal{M}) = \mathcal{M}_\odot
\end{equation}
Thus, $N_0$ is the number of solar masses created in a star formation burst. 

The IMF is assumed to be a power law with mass. This is justified because star formation proceeds through many orders of magnitude of densities and temperatures, so we expect the IMF is be a pretty featureless function. The power law must be chosen such that the IMF steepens with increasing mass. It is given by:

\begin{equation}
\xi  \propto \mathcal{M}^{-2.35}
\end{equation}
This power law, valid for $\mathcal{M} \geq 1\mathcal{M}_\odot$ is known as the Salpeter IMF.

The IMF can also be expressed in terms of the luminosity function:

\begin{equation}
\xi(\mathcal{M}) = \frac{dM}{d\mathcal{M}} \Phi_0 [M(\mathcal{M})]
\end{equation}
where M is absolute magnitude and $\mathcal{M}$ is mass. The function $M(\mathcal{M})$ can be found theoretically or observationally. It can be calculated using models of stellar atmospheres and main sequence stars of varying masses and metallicities. These models break down at $M(\mathcal{M}) \leq 0.6 M(\mathcal{M}_\odot)$ due to spectral deviation from black bodies. $M(\mathcal{M})$ can also be found using mass measurements of binary stars, although the data is generally sparse and noisy. 

\section{Malmquist Bias}

In any apparent magnitude-limited survey, intrinsically brighter objects will be overrepresented. The sample's true mean absolute magnitude is biased by the amount

\begin{equation}
\overline{\Delta M} = \left<M\right>_m - M_0
\end{equation}
where $\left<M\right>_m $ is the uncorrected mean absolute magnitude, and $M_0$ is the true mean absolute magnitude. 

The Malmquist bias $\overline{\Delta M}$ is calculated using $A(m)$, the total number of stars brighter than the apparent magnitude $m$, and $\sigma$, the dispersion in the absolute magnitude:
\begin{equation}
\overline{\Delta M} = \left<M\right>_m - M_0 = -\sigma^2 \frac{d ln A}{dm}.
\end{equation}
The function A(m) is defined as

\begin{equation}
A(m) = w\int^\infty_{-\infty} ds \Phi(M) s^2 \nu(s)
\end{equation}
where w is the solid angle and s is the distance. We probably won't have to use that expression on the qual, but it may be useful to note that $\Phi(M)$ is often approximated as a gaussian centered on $M_0$ with a standard deviation of $\sigma$. 

The Malmquist bias also allows for a more accurate calculation of the mean distance to a sample, given the distance modulus. For example, solving for distance in the equation $(m-M_m)-\overline{\Delta M} = 5log\left(\frac{D}{10} \right)$.

\subsection{G-dwarf problem}

The closed box model of chemical evolution predicts that one half of all stars near the sun have less than a third of the metallicity of the most metal rich stars.  Since the latter have metallicities comparable to that of the Sun, the closed box model implies that one half of solar neighborhood stars should have metallicities less than $\frac{1}{3}z_\odot$.  However, from observations only $2\%$ of disk F and G stars in the solar neighborhood have $z < -.25 z_\odot$.  This contradiction between the standard close-box model and observation is known as the G-dwarf problem.  From observations, stars are more metal rich than predicted.

\subsection{Distance Measures}

Moving cluster method:

\begin{equation}
D = \frac{-\theta}{\dot \theta}v_r
\end{equation}

Measuring the radial velocity of the cluster and the rate at which the cluster is expanding or shrinking allows you to find the distance to the cluster.

Using proper motions:

\begin{equation}
v_t = v_rtan(\theta)
\end{equation}
\begin{equation}
v_t = \mu d
\end{equation}
\begin{equation}
\mu = \frac{v_rtan(\theta)}{d}
\end{equation}
\begin{equation}
d = \frac{v_rtan(\theta)}{\mu}
\end{equation}

Baade-Wessilink Method:

This method requires an expanding/shrinking object, for example a variable star or a type Ia supernova.  The change in the radius of the object is given by:

\begin{equation}
r_1 - r_2 = v_r(t_1 - t_2)
\end{equation}

Next, use the flux measured at each time to determine the ratio of the radii at those times.

\begin{equation}
F_1 = \frac{4\pi r_1^2\sigma T_1^4}{4\pi d^2}
\end{equation}

\begin{equation}
\frac{r_1}{r_2} = \bigg(\frac{F_1T_2^4}{F_2T_1^4}\bigg)^{\frac{1}{2}}
\end{equation}

Now you can solve for the radii individually, and use the angular size of the object to get the distance.  


\subsection{Star Formation}

Observational methods to measure star formation:

1.  UV continuum - the UV spectrum is dominated by young massive stars, SFR scales linearly with luminosity.
	
	pros:  directly linked to young stellar population, wide range in redshift explorable.
	
	cons:  sensitivity to extinction and the IMF
	
2.  Recombination lines:  young stars emit UV radiation that ionizes atoms (mostly H) in nearby gas clouds.  Free electrons in the ionized gas then recombine with atoms.  They then drop to less excited states.  

	pros:  direct and highly sensitive
	
	cons:  extinction, IMF, uncertainty in gas distribution
	
3.  far-IR continuum:  UV radiation from young stars can be absorbed by interstellar dust and reemitted in the thermal infrared.

	pros:  sensitivity
	
	cons:  not direct, atmospheric absorption, distribution of dust, contribution of old stars




