\section{The Interstellar Medium}
\subsection{Questions}
\begin{enumerate}
\item Draw the cooling function for gas of solar metallicity, and describe the cooling
      mechanisms in each part of the curve. Explain the relevance of this function to multi-
      phase ISM models.
\item Draw a typical spectrum of an HII region, including both lines and continuum, and
      explain the major processes that give rise to each feature.
\item Explain quantitatively what determines the temperature of dust grains and their
      thermal emission spectrum. Give examples of astrophysical environments with different
      dust temperatures.
\end{enumerate}

\subsection{Phases of a Supernova}
\newthought{Anneila mentioned that} this was a fairly popular question on the quals, so
I believe a brief review of the phases of a supernova is relevant.  This focuses on the
supernova remant as opposed to the supernova mechanism itself.  Please edit this if I wrote
something that is wrong or could be explained better -- this is the point of using git.
\begin{enumerate}
    \item The free-expansion phase

    This phase is defined by the fact that the velocity of the ejecta is constant with time.
    The expanding mass of ejecta creates a shock that travels outwards through the ISM
    and a reverse shock that travels inwards through the ejecta.  The reverse shock has a finite
    amount of material to travel through and dies away.
    The outward traveling shock heats the ISM behind it to very high temperatures.
    Thermal bremsstrahlung is seen
    in X-rays, and synchrotron emission is seen in radio from ejected
    particles spiralling around $\mu$G magnetic fields.  This phase ends after 10s to 100s of
    years when the mass of the swept up ISM is comparable to the mass of the ejecta.
    Recall that high mass stars (the progenitors of core-collapse supernovae) typically
    drive large stellar winds.

    \item The adiabatic phase (or Sedov-Taylor phase)

    In this phase, the mass of the swept up ISM dominates the mass of the ejected material.
    However, the density of the material behind the shock is not yet large enough for cooling
    to be significant.  Therefore, the gas expands adiabatically.
    From dimensional analysis considerations, we can derive the Sedov-Taylor solution which says
    \begin{dmath}
        R \sim \left(\frac{Et^2}{\rho}\right)^{1/5}
    \end{dmath},
    where $R$ is the radius of the supernova remnant, $E$ is the energy released by the
    supernova, and $\rho$ is the density of the surrounding ISM.

    \item The radiative phase (or snowplough phase)
    \item Merger with the ISM
\end{enumerate}

