\section{The Interstellar Medium}

I added sections based on the list of important topics Allison gave us 
before the final. 

\subsection{Questions}
\begin{enumerate}
\item \textbf{Draw the cooling function for gas of solar metallicity, and describe the cooling
      mechanisms in each part of the curve. Explain the relevance of this function to multi-
      phase ISM models.}
\item \textbf{Draw a typical spectrum of an HII region, including both lines and continuum, and
      explain the major processes that give rise to each feature.}
\item \textbf{Explain quantitatively what determines the temperature of dust grains and their
      thermal emission spectrum. Give examples of astrophysical environments with different
      dust temperatures.}
\end{enumerate}

\subsection{Basic Things about ISM Phases}

The ISM phases are characterized by the state of hydrogen gas.

H$_2$ (molecular): $n \sim 200-10^6~{\rm cm}^{-3}$, $T \sim 10-500~{\rm K}$

HI (atomic): $n \sim 1-100~{\rm cm}^{-3}$, $T \sim 100-3000~{\rm K}$
 
HII (ionized): $n \sim 0.1-10^3~{\rm cm}^{-3}$, $T \sim 10^4-10^6~{\rm K}$
 
 Transitions between phases:
 
 Ionization of neutral hydrogen requires $13.6~{\rm eV}$, or photons with wavelength $\lambda < 912 \AA$, OR shocks that are faster than $50~{\rm km/s}$. In HII regions around hot (O,B) stars, the hydrogen is photoionized by the UV radiation from the star, and there are very sharp transitions between the HII, HI, and H$_2$ regions around the star.

\section{Heating and Cooling}

\section{Line Emission}

\section{HII regions}

\section{Dust}

\subsection{Phases of a Supernova}
\newthought{Anneila mentioned that} this was a fairly popular question on the quals, so
I believe a brief review of the phases of a supernova is relevant.  This focuses on the
supernova remnant as opposed to the supernova mechanism itself.  Please edit this if I wrote
something that is wrong or could be explained better -- this is the point of using git.
\begin{enumerate}
    \item The free-expansion phase

    This phase is defined by the fact that the velocity of the ejecta is constant with time. This is also known as ``homologous expansion" because the shape of the density profile remains the same. For every shell of the expanding ejecta, the distance from the source $r = vt$. The condition for being in this phase is that $M_{\rm ejecta} \ll M_{\rm swept}$, that is, the mass swept up by the ejecta from the ISM is not a significant fraction of the ejecta mass.
    The expanding mass of ejecta creates a shock that travels outwards through the ISM
    and a reverse shock that travels inwards through the ejecta.  The reverse shock has a finite
    amount of material to travel through and dies away.
    The outward traveling shock heats the ISM behind it to very high temperatures.
    Thermal bremsstrahlung is seen
    in X-rays, and synchrotron emission is seen in radio from ejected
    particles spiralling around $\mu$G magnetic fields.  This phase ends after 10s to 100s of
    years when the mass of the swept up from the ISM is comparable to the mass of the ejecta.
    Recall that high-mass stars (the progenitors of core-collapse supernovae) typically
    drive large stellar winds.

    \item The adiabatic phase (also called Sedov-Taylor or Blast Wave phase)

    In this phase, the mass of the swept up ISM dominates the mass of the ejected material ($M_{\rm swept} > M_{\rm ejecta}$).
    However, the density of the material behind the shock is not yet large enough for cooling
    to be significant.  Therefore, the gas expands adiabatically.
    From dimensional analysis considerations, we can derive the Sedov-Taylor solution which says
    \begin{dmath}
        R \sim \left(\frac{Et^2}{\rho}\right)^{1/5}
    \end{dmath},
    where $R$ is the radius of the supernova remnant, $E$ is the energy released by the
    supernova, and $\rho$ is the density of the surrounding ISM. The Sedov phase usually lasts about $10^4$ years.

    \item The radiative phase (or snowplow phase)
    
    This is also called the momentum-conserving phase. Consider a shell with mass $M_s$ and velocity $v_s$. The momentum $p_0 = M_s v_s$ is constant, giving the equation of motion
    \begin{equation}
    p_0 - \biggl(\frac{4}{3}\pi R^3 \rho_0 \biggr) \dot{R}\,\, ,
    \end{equation}
    which can be integrated to show that
    \begin{equation}
    R \propto t^{1/4}\,\,.
    \end{equation}
    \item Merger with the ISM
    
    Occurs when $v_s = c_s$ is the sound speed.
\end{enumerate}

\subsection{The Fluid Equations}
The following are the ideal or ``inviscid" fluid equations.

Mass conservation:
\begin{equation}
\frac{\partial \rho}{\partial t} + \vec{\nabla} \cdot \rho \vec{v} = 0
\end{equation}

Momentum conservation:
\begin{equation}
\frac{\partial}{\partial t} (\rho \vec{v}) + \vec{\nabla}\cdot (\rho \vec{v} \vec{v}) = -\vec{\nabla} P + \rho \vec{f}
\end{equation}

Energy conservation:
\begin{equation}
\frac{\partial}{\partial t} \biggl( \frac{1}{2} \rho v^2 + \rho \epsilon \biggr) + \vec{\nabla}\biggl[\biggl(\frac{1}{2}\rho v^2 + \rho \epsilon \biggr) + P \vec{v} \biggr] = \rho \vec{f} \cdot \vec{v}\,\,.
\end{equation}
The right-hand side of this equation is the work done by external forces (other people should feel free to add to this section or add different forms of the equations). Also, the Lagrangian or ``comoving" time derivative is

\begin{equation}
\frac{d}{dt} = \frac{\partial}{\partial t} + (\vec{v} \cdot \vec{\nabla})\,\, ,
\end{equation}
where $\frac{\partial}{\partial t}$ is the Eulerian time derivative, not comoving with the fluid.

It's probably good to know basically how to do perturbation theory on these equations, which you can do by saying $\rho \rightarrow \rho_0 + \delta \rho$ and the same with $P$ and $v$ (assume $v_0 = 0$), then killing all the terms that are second-order in $\delta$ quantities.

It's also a good idea to know Bernoulli's equation:
\begin{equation}
(\vec{v}\cdot \vec{\nabla})\biggl(\frac{1}{2} v^2 + h + \phi \biggr) = 0\,\, .
\end{equation}
or
\begin{equation}
\frac{1}{2} v^2 + h + \phi = {\rm constant}
\end{equation}
along every streamline. The first term is kinetic energy, the second is gravitational potential energy, and and third is thermal energy $+$ work that can be done.

\section{Shocks}

\section{Magnetic Fields}

\section{Star Formation}

\section{ISM in Other Galaxies}
