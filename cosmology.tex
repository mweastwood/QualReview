\section{Cosmology}
\subsection{Questions}
\begin{enumerate}
\item Put on a timeline, and describe the principal events in the thermal history of the
universe, from $kT = 10$ TeV to $kT = 0.1$ eV.
\item Give a semi-quantitative discussion of the connection between flucuations of the
cosmic microwave background on angular scales of arcminutes to degrees, and the
baryonic structures (galaxies, clusters, correlations of galaxies) observed in the local
universe, redshift $z < 0.5$.
\item Which elements/isotopes are produced in Big Bang Nucleosynthesis and in what
quantities? Explain qualitatively how the yield of each depends on the cosmic baryon
density and why.
\end{enumerate}

\subsection{Distances}
The basic concept of distance is the distance ladder, which is basically 
different techniques used to measure distances farther and farther away.  
There is overlap between the different techniques, which allows the next 
technique to be calibrated by the previous one.  The only problem with this 
is errors and uncertainties at the bottom of the ladder propogate all the 
way up.  

The first step is parallaxes, which are reliable out to $\sim100$ pc or so.  As 
the Earth moves around the Sun, the apparent positions of stars will change 
slightly relative to more distant fixed background stars.  The distance 
to an object in parsecs is defined to be $\frac{1}{p}$ where $p$ is the 
parallax in arcseconds.  One useful trick to know for quickly converting 
between angular size and physical size- the physical size in AU of something 
$d$ parsecs away with an angular size of $\theta$ arcsec is just $\theta d$.  
So, for example, if a star and planet are $1$ pc away and are separated by 
$1$ arcsec, the planet is $1$ AU from the star.  Parallax is the only 
distance indicator that is not model dependent or reliant on some other 
calibration.  

Out to distances of $\sim1$ kpc, main sequence fitting can be used to find 
distances to main sequence stars.  If the spectral type of a star is measured, 
and the absolute magnitude of that spectral type is known from a closer 
star with a parallax distance, the difference between apparent and absolute 
magnitude gives the distance to the new star.  

This has been done for clusters of stars that contain Cepheid variables, 
leading to the next step in the distance ladder.  
